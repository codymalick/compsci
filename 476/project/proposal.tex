\documentclass[12pt,letterpaper,onecolumn,draftclsnofoot]{IEEEtran}
\usepackage[margin=0.75in]{geometry}
\usepackage{listings}
\usepackage{color}
\usepackage{longtable}
\usepackage{tabu}
\definecolor{dkgreen}{rgb}{0,0.6,0}
\definecolor{gray}{rgb}{0.5,0.5,0.5}
\definecolor{mauve}{rgb}{0.58,0,0.82}

\lstset{frame=tb,
  language=C,
  columns=flexible,
  numberstyle=\tiny\color{gray},
  keywordstyle=\color{blue},
  commentstyle=\color{dkgreen},
  stringstyle=\color{mauve},
  breaklines=true,
  breakatwhitespace=true,
  tabsize=4
}

\let\endtitlepage\relax

\begin{document}
\begin{titlepage}
  \title{CS 476 - Winter 2017\\Cloud SDN and Load Balancer}
  \author{Cody Malick, Undergraduate CS\\ Dan Stoyer, Undergraduate CS \\
  	Andy Garcia, Undergraduate ECE and CS}
  \date{January 20, 2017}
  \maketitle
  \vspace*{1cm}
\end{titlepage}

\section{Project Description}
For this project, we have chosen to work on the "traffic load balancing through
software-defined networking" project, with the twist that we will use a cloud
platform. The project has three major components, the first being the creation
of some cloud virtual machine on a service such as AWS or the Google Cloud
Platform. Once the VM is set up, we can set up a local virtual network using
Mininet. Lastly, we will set up a simple load balancing service on top of the
SDN service.

\section{Motivation}
These problems are real problems in industry, and are very applicable to our
team's education and future employment. Cloud technologies are becoming more
and more relevant with the rise of large-scale cloud platforms in enterprise
computer. Learning about basic cloud technologies, SDN, and load balancing will
all come in handy in the future. 

\section{Solution Method}
For the first part of the project, we plan on using a free, enterprise based
platform, such as Amazon Web Services Free-Tier, or Google Cloud Platform's
free trial, to initialize a Virtual Machine online. After the VM is created,
we will use Mininet to create a virtual network layer inside the VM. Using that
network layer, we can then spin up a few web servers to act as our individual
nodes. Then, using the established framework, add a simple load balancing
service on top of the SDN. We will use a simple load balancing algorithm, such
as round robin.

\section{Evaluation Method}
We will evaluate the network performance using third party tools such as 
Wireshark, or build in Mininet utilities like \textit{iperf}.\cite{design} We
can also do basic analysis and evaluation of the performance drop of a SDN vs
non-SDN. The load balancer performance can also be evaluated to see how well it
distributes load over the Mininet Hosts.

\section{Execution Plan}
The project goals will be divided into three parts:

\begin{enumerate}
\item Environment Construction, setting up the testing environment, SDN, and 
	load balancing service. Cody will work on setting the VM up, as well as
	setting up the SDN with Mininet. Dan will focus on the SDN and load
	balancer. Andy will work with Dan on the SDN and load balancer.
\item Testing, gathering statistics on Mininet hosts as to network performance,
	and load balancing effectiveness. This will be split up throughout the
	team when the environment has been set up. It will be a group effort
	to gather data on different aspects of the project, such as latency, 
	and load balancer effectiveness.
\item Evaluation, writing up report on test results, and evaluation of results.
	This will, again, be split up throughout the team to use statistical
	analysis to understand the results of the test data.
\end{enumerate}

Dividing the project into these parts will give us a clear indication of where
we are in terms of progress, and a good idea if we've hit a roadblock on any
individual piece. 

\section{Bibliography}
\bibliographystyle{IEEEtran}
\bibliography{proposal}

\end{document}
