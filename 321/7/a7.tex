\documentclass[10pt,letterpaper]{article}

\usepackage[margin=0.75in]{geometry}
\usepackage{tikz}
\usepackage{amsmath}
\begin{document}

  \title{CS 321, Assignment 7}
  \author{Cody Malick\\
  \texttt{malickc@oregonstate.edu}}
  \date{\today}
  \maketitle

\section{}
\subsection*{a}
1. Adversary picks a number $p \geq 0$\\
2. I pick a string $s \in A$ \\
3. Adversary breaks $s$ into $s = uvwxy$, such that $|vwx| \leq p$ and $|vx| > 0$\\
4. I pick a number $i \geq 0$. If $uv^iwx^iy \notin A$, then I win

\subsection*{b}
1. Adversary picks a number $p \geq 0$\\
2. I pick a string $s \in A$ \\
3. Adversary breaks $s$ into $s = uvwxy$, such that $|vwx| \leq p$ and $|vx| > 0$\\
4. I pick a number $i \geq 0$. If $uv^iwx^iy \notin A$, then I win

\section{}
The tape head can move left and right freely along the tape, while being able
to read and write freely on a theoretically limitless tape. The tape is our way
of showing memory. If the tape head needs to read a value at a specific index,
it moves to that index, reads what's on the tape, and updates it as necessary.
Using the ability to read, write, and move freely, we can implement algorithms
to recognize and "compute" yes or no answers to problems. \\

\begin{center}
	\begin{tabular}{ | p{6cm} | p{6cm} | p{6cm} | }
		\hline
		\multicolumn{3}{|c|}{Turing Machine for $\{a^nb^mc^{nm}\}$} \\
		\hline
		If in state: & reading: & do: \\ \hline
		look for an a & test \newline testt& \\ \hline
		look for a b & & \\ \hline
		look for a c & & \\ \hline


\end{tabular}
\end{center}
\end{document}
