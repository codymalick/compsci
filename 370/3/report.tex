\documentclass[10pt,letterpaper]{article}

\usepackage[margin=0.75in]{geometry}
\usepackage{tikz}
\usepackage{graphicx}

\usepackage{listings}
\usepackage{color}
\usepackage{tabu}
\definecolor{dkgreen}{rgb}{0,0.6,0}
\definecolor{gray}{rgb}{0.5,0.5,0.5}
\definecolor{mauve}{rgb}{0.58,0,0.82}

\lstset{frame=tb,
  language=Bash,
  columns=flexible,
  numberstyle=\tiny\color{gray},
  keywordstyle=\color{blue},
  commentstyle=\color{dkgreen},
  stringstyle=\color{mauve},
  breaklines=true,
  breakatwhitespace=true,
  tabsize=4
}

\graphicspath{{img/}}

\begin{document}

\title{CS 370, Assignment 3 Report}
\author{Cody Malick\\
\texttt{malickc@oregonstate.edu}}
\date{\today}
\maketitle

\section*{Question 1}
Steps taken:\\
1. used chmod u-s /usr/bin/passwd
2. Tested passwd again to see what permissions were needed
Error received: passwd: Authentication token manipulation error
3. Researched what capabilities passwd needs under the posix capabilities system
4. Capbilities needed: cap\_chown, cap\_dac\_override, cap\_fowner, all needed to be
set to ep

From the man pages, the following capabilities grant the following permissions:

cap\_chown: Make arbitrary changes to file UIDs and GIDs

cap\_dac\_override: Bypass file read, write, and execute permission checks.

cap\_fowner: Bypass permission checks on operations that normally require the
filesystem UID of the process to match the UID of the file. Set
extended file attributes on arbitrary files. Set Access Control Lists
on arbitrary files.

\section*{Question 2}

cap\_dac\_read\_search: Bypass file read permissions checks and directory read and
	execute permission checks.

We can demonstrate this capability by removing the `ls' command's set-UID capability
using:
\begin{lstlisting}[caption=Removing set-UID capability on ls]
root@code-virtual-machine: chmod u-s /bin/ls
\end{lstlisting}

Then, we can test out the lack of set-UID capability by running ls on a directory
that requires root access, such as /root:

\begin{lstlisting}
code@code-virtual-machine:~/git/compsci/370/3\$ ls /root
ls: cannot open directory '/root': Permission denied
\end{lstlisting}

Now, we can add the cap\_dac\_read\_search capability to ls:
\begin{lstlisting}
code@code-virtual-machine:~/git/compsci/370/3\$ sudo setcap cap_dac_read_search=ep /bin/ls
code@code-virtual-machine:~/git/compsci/370/3\$ ls -a /root
.  ..  .bash_history  .bashrc  .bzr.log  .cache  .profile
\end{lstlisting}

We've demonstrated that, using the cap\_dac\_read\_search capability, we can remove
the need for ls to run at an elevated level.

\section*{Question 3}

\section*{Question 4}
\section*{Question 5}
\section*{Question 6}


\end{document}



