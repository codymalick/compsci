\documentclass[10pt,letterpaper]{article}

\usepackage[margin=0.75in]{geometry}
\usepackage{tikz}
\usepackage{graphicx}

\usepackage{listings}
\usepackage{color}
\usepackage{tabu}
\definecolor{dkgreen}{rgb}{0,0.6,0}
\definecolor{gray}{rgb}{0.5,0.5,0.5}
\definecolor{mauve}{rgb}{0.58,0,0.82}

\lstset{frame=none,
  language=Bash,
  columns=flexible,
  numberstyle=\tiny\color{gray},
  keywordstyle=\color{blue},
  commentstyle=\color{dkgreen},
  stringstyle=\color{mauve},
  breaklines=true,
  breakatwhitespace=true,
  tabsize=4
}

\graphicspath{{img/}}

\begin{document}

\title{CS 370, Assignment 3 Report}
\author{Cody Malick\\
\texttt{malickc@oregonstate.edu}}
\date{\today}
\maketitle

\section*{Question 1}
Steps taken:\\
1. used chmod u-s /usr/bin/passwd
2. Tested passwd again to see what permissions were needed
Error received: passwd: Authentication token manipulation error
3. Researched what capabilities passwd needs under the posix capabilities system
4. Capbilities needed: cap\_chown, cap\_dac\_override, cap\_fowner, all needed to be
set to ep

From the man pages, the following capabilities grant the following permissions:

cap\_chown: Make arbitrary changes to file UIDs and GIDs

cap\_dac\_override: Bypass file read, write, and execute permission checks.

cap\_fowner: Bypass permission checks on operations that normally require the
filesystem UID of the process to match the UID of the file. Set
extended file attributes on arbitrary files. Set Access Control Lists
on arbitrary files.

\section*{Question 2}

cap\_dac\_read\_search: Bypass file read permissions checks and directory read and
	execute permission checks.

We can demonstrate this capability by removing the `ls' command's set-UID capability
using:
\begin{lstlisting}[caption=Removing set-UID capability on ls]
root@ubuntu: chmod u-s /bin/ls
\end{lstlisting}

Then, we can test out the lack of set-UID capability by running ls on a directory
that requires root access, such as /root:

\begin{lstlisting}
seed@ubuntu:~/ ls /root
ls: cannot open directory '/root': Permission denied
\end{lstlisting}

Now, we can add the cap\_dac\_read\_search capability to ls:
\begin{lstlisting}
seed@ubuntu:~/ sudo setcap cap_dac_read_search=ep /bin/ls
seed@ubuntu:~/ ls -a /root
.  ..  .bash_history  .bashrc  .bzr.log  .cache  .profile
\end{lstlisting}

We've demonstrated that, using the cap\_dac\_read\_search capability, we can remove
the need for ls to run at an elevated level.

\section*{Question 3}
After the capability was added, tests b,d, and e failed. Let's break down what
happened after the capability was added.
\subsection*{a}
After: Passed\\
Observations: This test just tries to open the file initially. It passes. 

\subsection*{b}
After: Failed\\
Observations: This test should have failed, as we call cap\_disable(), which
allows the process to temporarily disable the privilege.

\subsection*{c}
After: Passed\\
Observations: This test passes, as we have re-enabled the capability using
cap\_enable().

\subsection*{d}
After: Failed\\
Observations: This test fails because we drop the cabability completely using
cap\_drop(), making it unrecoverable.

\subsection*{e}
After: Failed\\
Observations: This test tries to reenable the capability after it was dropped.
The test fails because, once the capability has been deleted from the process,
it cannot be recovered soley by the process.

\section*{Question 4}
In ACL, we would have to assign elevated privileges to a certain person or
role in order to accomplish what we did here. Linux uses the `sudo' command
to make this relatively simple. While typing `sudo' is fairly easy and convenient,
it does not allow very fine grain control of the system. The more sudoers on a
given system, the more security vulnerabilities there are.

Capabilites, while less conveniant, are more fine grained in their control of
the system. Using capabilities properly would reduce the number of sudoers or
super users needed on the system.

\section*{Question 5}
The attacker would be able to use capability A if he managed to call the code
to reenable it. Disabling the capability is only temporary.
If the capability was completely dropped or deleted, however, he would not be
able to unless he was able to compromise the code before the capability was
dropped.

\section*{Question 6}


\end{document}



