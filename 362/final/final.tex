\documentclass[10pt,letterpaper]{article}

\usepackage[margin=0.75in]{geometry}
\usepackage{tikz}

\begin{document}

  \title{CS 362, Final}
  \author{Cody Malick\\
  \texttt{malickc@oregonstate.edu}}
  \date{\today}
  \maketitle

\section*{Problem 1}
	\subsection*{I}
	\subsubsection*{a}
		You would need one for every possible set of inputs, not taking
		into account bad inputs:
		\begin{itemize}
			\item foo(false,false,false)
			\item foo(true,false,false)
			\item foo(false,true,false)
			\item foo(true,true,false)
			\item foo(true,false,true)
			\item foo(false,true,true)
			\item foo(true,true,true)
		\end{itemize}

		7 total inputs. If, however, we needed to test all possible bad
		input, it would be a very large set of tests including other
		data types, nulls, etc. The total number is not easily
		countable.

	\subsubsection*{b}
		Functionality partitioning would not be possible given only the
		prototype of the function. In order to perform functionality
		based partitioning, you need to be able to analyse the behavior
		of the function itself.

	\subsection*{II}
		\subsubsection*{a}
			To exhaustively test the function, you would need to
			pass all possible values of $a$, $b$, and $c$ to the
			function (as stated in the answer to question I(a)),
			totaling seven cases, not including all possible bad
			input, null or otherwise.\\
			More efficiently, you can test all functionality by
			looking at all the possible branches in the function:
			\begin{itemize}
				\item a==b
				\item a!=b
				\item b==c
				\item b!=c
			\end{itemize}
			Totaling four primary cases.

		\subsubsection*{b}
			You could perform interface-based partitioning, but you
			would end up with tests cases that would not vary. For
			example, testing a==c:\\\\
				biz(true,false,true)\\\\
			Would not change the behavior of the function.

	\subsection*{III}
	\subsubsection*{a}
		To \textit{exhaustively} test the function would require
		spanning all possible values of $a$, which would be equal to
		$2^{32}$ possible values. This does not include testing other
		bad values, null, wrong types, etc.

	\subsubsection*{b}
		The most intuitive way to partition this function with
		functionality based testing would be to look at the possible
		branches, while $k < a$, and while $k > a$. This gives us two
		possible branches that we should test functionality on.

\section*{Problem 4}
	This function could be randomly tested with a two pronged approach:\\\\
	First, test the directory functionality by randomly picking a real
	directory name and passing it into the function, or a randomly generated
	string, containing characters that couldn't be real directories, and
	checking the results for \textit{FileNotFoundException}.
	Second, for each directory passed, real and fake, get a list of real
	files and create a list of fake files, and randomly pull from either,
	checking to see if the result is correct or not.\\\\
	Enough runs through this scheme would test all possibilities:
	\begin{itemize}
		\item fake directories, fake files
		\item real directories, fake files
		\item fake directories, real files
		\item real directories, real files. 
	\end{itemize}

\end{document}
