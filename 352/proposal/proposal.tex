\documentclass[10pt]{article}
\usepackage[margin=1in]{geometry}

\begin{document}
\title{CS 352, Project Proposal\\ Twitch Stream Metadata Website}
\author{Cody Malick\\
\texttt{malickc@oregonstate.edu}}
\date{\today}
\maketitle

\section*{Project}
Twitch.tv is one of the most popular sites on the web. It is a free platform
that allows video game players to share their experiences with viewers. Hundreds
of thousands of people visit the site each day. While the site itself is
very well designed, there is no way to easily get metadata or statistics about 
an individual stream. For example, a streamer might wanted to know how well his stream has
done over a few months, or even compare those statistics to other Twitch streams.
This data is accessible through the Twitch API, and could be displayed using 
intuitive graphs and interactive charts.

\section{Screens}
\subsection{Home Screen}
The home screen would display a random high traffic stream's metadata. Data
included would be the name of the stream, current viewers, and popular games.
Examples of other data include:
\begin{itemize}
	\item Unique visitors per day
	\item Uptime of stream on average
	\item Number of subscribers over time
	\item Number of followers over time
	\item A graph of viewers throughout the length of the stream
	\item Views per game (if the streamer switched games)
\end{itemize}

Useful or interesting information from the graphs could be extracted. For example,
when is a stream most popular? Which game draws the most viewers? 

\subsection{Tabs}
While a homescreen could contain important information, each screen could have
tabs that contain other extraneous information. An example subtab could be
a zoomed in version of a graph on the home page. 

\subsection{Search}
There would need to be a search screen allowing a user to find the data about
a particular stream. This would be a simple search box, either as a seperate 
screen, or build into the navigation bar at the top of the website.

\section{Challenges}
While a site like this would be very fun and interesting, there are a number
of usability and design challenges that would have to be examined. While data
visualization is fun, how much data should be shown on one screen? Too much
data would overwhelm the user, while too little would cause the user problems
clicking through tabs. 

\end{document}
