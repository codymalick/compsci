\documentclass[10pt,letterpaper]{article}

\usepackage[margin=0.75in]{geometry}
\usepackage{tikz}
\usepackage{amsfonts}
\usepackage{xspace}
\usepackage{graphicx}
\usepackage{mathtools}
\usepackage{amssymb}
\usepackage{listings}
\usepackage{amsmath}
\usepackage{color}
\usepackage{empheq}
\usepackage{bm}
\definecolor{myblue}{rgb}{.8, .8, 1}
\newlength\mytemplen
\newsavebox\mytempbox

\makeatletter
\newcommand\mybluebox{%
\@ifnextchar[%]
	{\@mybluebox}%
{\@mybluebox[0pt]}}

\def\@mybluebox[#1]{%
	\@ifnextchar[%]
		{\@@mybluebox[#1]}%
	{\@@mybluebox[#1][0pt]}}

\def\@@mybluebox[#1][#2]#3{
	\sbox\mytempbox{#3}%
	\mytemplen\ht\mytempbox
	\advance\mytemplen #1\relax
	\ht\mytempbox\mytemplen
	\mytemplen\dp\mytempbox
	\advance\mytemplen #2\relax
	\dp\mytempbox\mytemplen
\colorbox{myblue}{\hspace{1em}\usebox{\mytempbox}\hspace{3em}}}

\makeatother

\begin{document}
\title{CS 427, Assignment 3}
\author{Cody Malick\\
\texttt{malickc@oregonstate.edu}}
\date{\today}
\maketitle

\section{}
	\subsection*{a}
	It is fairly easy to show that this function is insecure by using a
	chosen plaintex attack. Here is a quick example. Suppose you have some
	calling function $\mathcal{F}$:

	\begin{empheq}[box={\mybluebox[3pt]}]{equation*}
		\begin{split}
			\mathcal{F}():\\
			& \quad m_1\ \leftarrow\ \{0,1\}^{\lambda}\\
			& \quad m_2\ \leftarrow\ \{0,1\}^{\lambda}\\
			& \quad k\ := Keygen() \\
			& \quad // Split the resulting cipher text into two parts
			& \quad c_1,c_2\ :=\ H(k,m_1||m_2) \\
			& \quad if(\ m_2\ ==\ c_1) {
			& \qquad return true \\
			& \quad } \\
			& \quad return false
			& \quad return\ H(s)\\
		\end{split}
	\end{empheq}

	Examining this attack, we can see that $\mathcal{F}$ will return true
	with a probability of 1. The PRF provides no encryption for the first
	half of the plain text. If $H()$ were secure, we should not be able to
	differentiate between $H()$ and some secure $PRP$.

	\subsection*{b}
	While this function is slightly more secure than the last, it can also
	be attacked with a chosen plaintext attack, but we have to be a little
	smarter about how we extract information. If we feed the two round cipher
	the same plaintext, we can reduce the output range by $\frac{1}{2}$:

\end{document}
