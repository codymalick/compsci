\documentclass[10pt,letterpaper]{article}

\usepackage[margin=0.75in]{geometry}
\usepackage{tikz}
\usepackage{amsfonts}
\usepackage{xspace}
\usepackage{graphicx}
\usepackage{mathtools}
\usepackage{amssymb}
\usepackage{listings}
\usepackage{amsmath}
\usepackage{color}
\usepackage{empheq}
\usepackage{bm}
\definecolor{myblue}{rgb}{.8, .8, 1}
\newlength\mytemplen
\newsavebox\mytempbox

\makeatletter
\newcommand\mybluebox{%
\@ifnextchar[%]
	{\@mybluebox}%
{\@mybluebox[0pt]}}

\def\@mybluebox[#1]{%
	\@ifnextchar[%]
		{\@@mybluebox[#1]}%
	{\@@mybluebox[#1][0pt]}}

\def\@@mybluebox[#1][#2]#3{
	\sbox\mytempbox{#3}%
	\mytemplen\ht\mytempbox
	\advance\mytemplen #1\relax
	\ht\mytempbox\mytemplen
	\mytemplen\dp\mytempbox
	\advance\mytemplen #2\relax
	\dp\mytempbox\mytemplen
\colorbox{myblue}{\hspace{1em}\usebox{\mytempbox}\hspace{3em}}}

\makeatother

\begin{document}
\title{CS 427, Assignment 3}
\author{Cody Malick\\
\texttt{malickc@oregonstate.edu}}
\date{\today}
\maketitle

\section{}
	\subsection*{a}
	It is fairly easy to show that this function is insecure by using a
	chosen plaintex attack. Here is a quick example. Suppose you have some
	calling function $\mathcal{F}$, where $H$ is either the function we're attacking
	or a function that is a secure PRP, that returns a uniformly
	distributed encrypted value:

	\begin{empheq}[box={\mybluebox[3pt]}]{equation*}
		\begin{split}
			\mathcal{F}():\\
			& \quad m_1\ \leftarrow\ \{0,1\}^{\lambda}\\
			& \quad m_2\ \leftarrow\ \{0,1\}^{\lambda}\\
			& \quad k\ := Keygen() \\
			& \quad //Split\ the\ resulting\ cipher\ text\ into\ two\ parts\\
			& \quad c_1,c_2\ :=\ H(k,m_1||m_2) \\
			& \quad if(\ m_2\ ==\ c_1) \{\\
			& \qquad return\ true \\
			& \quad \} \\
			& \quad return\ false\\
		\end{split}
	\end{empheq}

	Examining this attack, we can see that $\mathcal{F}$ will return true
	with a probability of 1 for the function $H$ we are attacking, and will
	almost never return true for a properly ecrypted value.
	The PRF provides no encryption for the first
	half of the plain text. If $H()$ were secure, we should not be able to
	differentiate between $H()$ and some secure $PRP$.

	\subsection*{b}
	While this function is slightly more secure than the last, it can also
	be attacked with a chosen plaintext attack, but we have to be a little
	smarter about how we extract information. We can gain information about
	the plaintext by setting the first half of two messages to $m_1$, and 
	the later half of either to any other string. In this case, $m_2$ and
	$m_3$. 
	
\section{}
	To show that $F'$ is a secure PRF, we have to show that the output, which
	in this case would be uniformly random, is indistinguishable from a 
	secure PRF. We can show this by transforming $F'$ to look have output
	like some other function, $F$. Starting with our base definition of
	$F'$:

	\begin{empheq}[box={\mybluebox[3pt]}]{equation*}
		\begin{split}
			F'(k,r):\\
			& \quad return\ G(F(k,r))\\
		\end{split}
	\end{empheq}

	We first can expand all the operations:


	\begin{empheq}[box={\mybluebox[3pt]}]{equation*}
		\begin{split}
			F'(k,r):\\
			& \quad f\ :=\ F(k,r)\\
			& \quad g\ :=\ G(f)\\
			& \quad return\ g\\
		\end{split}
	\end{empheq}

	This does not change the calling function as we have not changed the
	fundamental behavior of the function. Next, we can pull $F$ out into a
	calling function $\mathcal{F}$, and substitute $\mathcal{F}$ for
	it's resulting output, a uniformly random distribution:


	\begin{empheq}[box={\mybluebox[3pt]}]{equation*}
		\begin{split}
			F'(k,r):\\
			& \quad f\ :=\ \mathcal{F}(k,r)\\
			& \quad g\ :=\ G(f)\\
			& \quad return\ g\\
		\end{split}
	\end{empheq}

	
	\begin{empheq}[box={\mybluebox[3pt]}]{equation*}
		\begin{split}
			\mathcal{F}(k,r):\\
			& \quad return x \leftarrow \{0,1\}^\lambda\\
		\end{split}
	\end{empheq}

	Substitute the results of the new function $\mathcal{F}$:

	\begin{empheq}[box={\mybluebox[3pt]}]{equation*}
		\begin{split}
			F'(k,r):\\
			& \quad f\ \leftarrow\ \{0,1\}^\lambda\\
			& \quad g\ :=\ G(f)\\
			& \quad return\ g\\
		\end{split}
	\end{empheq}

	We can do this, because we have assumed that $F$ is a secure PRF, which
	outputs a random number picked from a uniformly random distribution.
	Next, we can perform the same proceedure with $G$, but we must show why
	it works for results of length $2\lambda$. $G$ is pulled out into function
	$\mathcal{G}$, and then is substitued:

	\begin{empheq}[box={\mybluebox[3pt]}]{equation*}
		\begin{split}
			F'(k,r):\\
			& \quad f\ :=\ \{0,1\}^\lambda\\
			& \quad g\ :=\ \mathcal{G}(f)\\
			& \quad return\ g\\
		\end{split}
	\end{empheq}

	\begin{empheq}[box={\mybluebox[3pt]}]{equation*}
		\begin{split}
			\mathcal{G}(x):\\
			& \quad x_1\ :=\ first\ half\ of\ x\\
			& \quad x_2\ :=\ second\ half\ of\ x\\
			& \quad c_1\ :=\ \mathcal{F}(x_1) \\
			& \quad c_2\ :=\ \mathcal{F}(x_2) \\
			& \quad return c_1 || c_2\\
			& \quad return\ g\\
		\end{split}
	\end{empheq}

	With this outline complete, we can then substitute the previous result
	of $\mathcal{F}$, specifically its random output:

	\begin{empheq}[box={\mybluebox[3pt]}]{equation*}
		\begin{split}
			\mathcal{G}(x):\\
			& \quad c_1\ \leftarrow\ \{0,1\}^\lambda\\
			& \quad c_2\ \leftarrow\ \{0,1\}^\lambda\\
			& \quad return\ c_1 || c_2\\
		\end{split}
	\end{empheq}

	This result can be rewritten as a random value of length $2\lambda$:


	\begin{empheq}[box={\mybluebox[3pt]}]{equation*}
		\begin{split}
			F'(k,r):\\
			& \quad f\ \leftarrow\ \{0,1\}^\lambda\\
			& \quad g\ \leftarrow\ \{0,1\}^{2\lambda}\\
			& \quad return\ g\\
		\end{split}
	\end{empheq}

	This is the desired result, as the output is a uniformly random distribution
	of length $2\lambda$. This result is indistinguishable from a secure
	PRF, and that completes the proof.
\section{}
	Given that $F$ is a PRP, then it must, by definition, have an inverse,
	$F^{-1}$. The decryption algorith for this setup would then be:

	\begin{empheq}[box={\mybluebox[3pt]}]{equation*}
		\begin{split}
			DEC(k,(x,y)):\\
			& \quad s_2\ :=\ F^{-1}(k,y)\\
			& \quad s_1\ :=\ F^{-1}(k,x)\\
			& \quad m\ :=\ s_2 \oplus s_1\\
			& \quad return\ m\\
		\end{split}
	\end{empheq}

	
\end{document}
