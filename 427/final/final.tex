\documentclass[10pt]{article}
%%%%%%%%%%%%%%%%%%%%%%%
% Mike's stuff

%% Sample LaTeX for CS427/519
%% Mike Rosulek
%% last update 2017-01-23

\usepackage{xspace,graphicx,amsmath,amssymb,xcolor}
\usepackage[margin=1in]{geometry}
\usepackage{graphicx}
\graphicspath{{images/}}
%% operators

\newcommand{\pct}{\mathbin{\%}}
% makes ":=" aligned better
\usepackage{mathtools}
\mathtoolsset{centercolon}

% indistinguishability operator
% http://tex.stackexchange.com/questions/22168/triple-approx-and-triple-approx-with-a-straight-middle-line
\newcommand{\indist}{  \mathrel{\vcenter{\offinterlineskip
\hbox{$\sim$}\vskip-.35ex\hbox{$\sim$}\vskip-.35ex\hbox{$\sim$}}}}
\renewcommand{\cong}{\indist}

\newcommand{\K}{\mathcal{K}}
\newcommand{\M}{\mathcal{M}}
\newcommand{\C}{\mathcal{C}}
\newcommand{\Z}{\mathbb{Z}}

\newcommand{\Enc}{\text{\sf Enc}}
\newcommand{\Dec}{\text{\sf Dec}}
\newcommand{\KeyGen}{\text{\sf KeyGen}}

% fancy script L
\usepackage[mathscr]{euscript}
\renewcommand{\L}{\ensuremath{\mathscr{L}}\xspace}
\newcommand{\lib}[1]{\ensuremath{\L_{\textsf{#1}}}\xspace}

\newcommand{\myterm}[1]{\ensuremath{\text{#1}}\xspace}
\newcommand{\bias}{\myterm{bias}}
\newcommand{\link}{\diamond}
\newcommand{\subname}[1]{\ensuremath{\textsc{#1}}\xspace}

%% colors
\definecolor{highlightcolor}{HTML}{F5F5A4}
\definecolor{highlighttextcolor}{HTML}{000000}
\definecolor{bitcolor}{HTML}{a91616}

%%% boxes for writing libraries/constructions
\usepackage{varwidth}

\newcommand{\codebox}[1]{%
	\begin{varwidth}{\linewidth}%
		\begin{tabbing}%
			~~~\=\quad\=\quad\=\quad\=\kill % initialize tabstops
			#1
		\end{tabbing}%
	\end{varwidth}%
}
\newcommand{\titlecodebox}[2]{%
	\fboxsep=0pt%
	\fcolorbox{black}{black!10}{%
		\begin{varwidth}{\linewidth}%
			\centering%
			\fboxsep=3pt%
			\colorbox{black!10}{#1} \\
			\colorbox{white}{\codebox{#2}}%
		\end{varwidth}%
	}
}
\newcommand{\fcodebox}[1]{%
	\framebox{\codebox{#1}}%
}
\newcommand{\hlcodebox}[1]{%
	\fcolorbox{black}{highlightcolor}{\codebox{#1}}%
}
\newcommand{\hltitlecodebox}[2]{%
	\fboxsep=0pt%
	\fcolorbox{black}{black!15!highlightcolor}{%
		\begin{varwidth}{\linewidth}%
			\centering%
			\fboxsep=3pt%
			\colorbox{black!15!highlightcolor}{\color{highlighttextcolor}#1} \\
			\colorbox{highlightcolor}{\color{highlighttextcolor}\codebox{#2}}%
		\end{varwidth}%
	}
}


%% highlighting
\newcommand{\basehighlight}[1]{\colorbox{highlightcolor}{\color{highlighttextcolor}#1}}
\newcommand{\mathhighlight}[1]{\basehighlight{$#1$}}
\newcommand{\highlight}[1]{\raisebox{0pt}[-\fboxsep][-\fboxsep]{\basehighlight{#1}}}
\newcommand{\highlightline}[1]{%\raisebox{0pt}[-\fboxsep][-\fboxsep]{
	\hspace*{-\fboxsep}\basehighlight{#1}%
	%}
}

%% bits
\newcommand{\bit}[1]{\textcolor{bitcolor}{\texttt{\upshape #1}}}
\newcommand{\bits}{\{\bit0,\bit1\}}



%%%%%%%%%%%%%%%%%%%%%%%

\begin{document}
\title{CS 427, Final Project\\ POODLE\\ Padding Oracle on Downgraded Legacy
Encryption}
\author{Cody Malick\\
\texttt{malickc@oregonstate.edu}}
\date{\today}
\maketitle
\vspace{4cm}

\begin{abstract}
	This paper outlines the POODLE exploit. It gives details on how it
	allows an adversary to repeatedly query a server using SSL version 3.0,
	and eventually decrypt a block one byte at a time. The exploit was
	first published by the
	Google Security Team on October 14, 2014 by Bodo Moller, Thai Duong,
	and Kryzsztof Kotowicz. While SSL version 3.0 is quite old, it is still
	in use today for browser backward compatibility. Later, an updated
	version of the POODLE exploit was published showing successful attacks
	against TLS.

\end{abstract}

\clearpage

\section*{Introduction}
POODLE, or Padding Oracle On Downgraded Legacy Encryption, 
is an attack on Secure Socket Layer version 3.0 that was made public
October of 2014. It was announced on the Google Security Blog, published by
Bodo Moller, and co-published by Thai Duong, and Kryzsztof Kotowicz. It was
discovered while the good folks at Google Security Team were investigating
vulnerabilities in Chrome. I am of the opinion that the Google Security Team
published the vulnerability in an ethical fashion. It is apparent that they had
not known about the vulnerability for a long period as they moved quickly to
publish their results. The blog post mentions that they had not yet even
patched Google's own servers. \cite{googlesec}

POODLE allows an attacker targeting SSL 3.0 to decrypt one byte of an encrypted
SSL block one out of two-hundred fifty-six attempts. While this is an average,
being able to decrypt a byte of an encrypted block that quickly is quite the
security flaw.\cite{POODLE} It was later found that the vulnerability extended
to certain implementations of TLS version 1.0. Certain implementations of TLS in
popular software did not correctly check the padding after decryption. \cite{POODLEBites}

The goal of SSLv3 was to encrypt and provide secrecy and security for
communication between a web browser and a web server. This is accomplished
through an initial secret-sharing hand shake, computing a MAC from the shared
secret, and using AES CBC mode to encrypt and decrypt messages.\cite{SSLv3}
While the original goal of the SSLv3 was to provided secrecy, authenticity, and
integrity, it has since be antiquated by newer protocols due to design flaws
exploited by attacks such as POODLE.

POODLE is executed using a padding oracle attack. A padding oracle attack, simply
described, is abusing the fact that a web server will throw an error on a valid
ciphertext if the padding is bad. This allows an attacker to verify whether or
not an arbitrary ciphertext has valid padding. Furthermore, with this information,
and adversary can decrypt the arbitrary ciphertexts!\cite{rosulek} This would
allow an attacker access to sensitive information in a web browser such as
cookies or cache data.

\section*{Context}
To completely understand the attack, the context should first be described.
SSL version 3.0 is quite dated, and should for all intents and purposes be 
completely retired. Modern browsers, however, use SSLv3 as a fallback for
compatibility purposes. The browser will first try to connect to the web server
using TLS version 1.2, the de facto standard for web communication. If that fails,
it is default behavior in browsers to fall back to earlier standards, such as SSLv3.\cite{POODLE}

While browsers continue to support SSLv3, this attack will continue to be an
issue. A savvy attacker with the proper resources could catch web requests with
a man-in-the-middle attack, and prevent all non-SSLv3 requests from being transferred.
This would force the use of SSLv3 if it is supported by the web server.\cite{MozillaPOODLE}

\subsection*{Man-in-the-Middle}
Man-in-the-middle attacks are fairy simple in concept, but harder to execute.
The attack is executed by intercepting a communication between two hosts, and
reading or altering the information passed between them.\cite{mttm} While the
specifics are not important to understanding POODLE, it is important to know
that an attacker with a setup like this can read, alter, and inject completely
new data into a communication.

\subsection*{Padding}
The primary point of vulnerability that the attack exploits is in the padding
of the encrypted message. Padding in SSLv3 is accomplished by measuring the 
difference between the last block size, and the required size of the block. In
this case, the block size must be a multiple of sixteen. In this case, the AES
CBC mode block size is sixteen bytes, sixty-four bits. The SSLv3 implementation
then measures the difference $d$, and fills the space $d-1$ with random byte
values. The last byte is then filled with the size of the padding. The padding
can be zero, or a maximum of sixty-three bytes.\cite{SSLv3} It is possible for
an entire encryption block to be only padding.

When attempting to decrypt, the algorithm will check the last byte for the size of
the padding. The major issue with this becomes apparent here. Because the padding
of SSLv3 is nondeterministic, and the padding is not covered by the message
authentication code provided, any value can be provided in the place of the
padding byte.\cite{POODLE} More on this in the 'exploit' section.

\subsection*{AES CBC Mode}
Cipherblock Chaining mode is one of the most popular modes for AES operation.
It provides chosen plaintext attack security, and non-deterministic encryption
using an initialization vector. Here is a diagram from the class textbook that
illustrates how this process works:\cite{rosulek}

\includegraphics[scale=0.5]{cbcmode}

Of particular importance to notice for this attack is the formula used to decrypt
a given ciphertext block:\\

\noindent $m_i\ :=\ F^{-1}(k,c_i)\ \oplus\ c_{i-1}$\\

This formula will come up again in the exploit section. While understanding the
underlying encryption algorithm is important to understanding the exploit, the
primary point of vulnerability is the padding. The CBC encryption used in SSLv3
is sound. 

\section*{The Exploit}
The exploit is accomplished by executing the following steps:

\subsubsection*{SSL Version 3.0 Fallback}
While not required, being able to force a target back onto SSLv3 is often necessary.
The vast majority of modern web browsers and web servers use some version of TLS.
Per Mozilla, which has since completely disabled SSLv3 in Firefox, only about
0.3\% of all their requests used SSLv3.\cite{MozillaPOODLE} With that in mind,
it is likely that an adversary wishing to utilize this attack must first force
the web browser to fall back to SSLv3. This would typically be done during the
initial connection between the web browser and web server when they first try
to agree on a protocol to use. 

An adversary with a man-in-the-middle setup could disallow any connections
that were not SSLv3. If both the web browser and web server support SSLv3, they
would eventually fall back on using the protocol.\cite{POODLE} Once the connection
has been established, the stage is set for the attack.

\subsubsection*{Replacing the Padding Block}
Because SSLv3 does not verify the integrity of the padding or the padding byte
due to its nondeterministic nature, this is the point of attack. An adversary
who is able to intercept and alter traffic (our man-in-the-middle setup) can
replace the padding block with any other block.

More formally, given ciphertext blocks $C_1 ... C_n$, an initialization vector
$C_0$, and the block-cipher decryption $D_k$ using key $K$, the recipient of
the message would use the following process to decrypt to get $P$:\cite{POODLE}\\

\noindent $P_1 ... P_n, P_i\ =\ D_K(C_i)\ \oplus\ C_{i-1}$\\

The receiver would then check the padding on the end ($C_n$), and verify the
MAC. The attacker could replace $C_n$ with any other block of encrypted
ciphertext $C_i$ from the same message. Then the ciphertext will still be
accepted if the decryption of the last byte of block $C_i$ is equal to the
value of the padding byte. This is the padding oracle attack. \cite{POODLE}.

This means that the attacker knows that the last byte of the decrypted message
was in-fact equal to the value of the padding byte. The probability of this
attack being successful turns out to be extremely probable. With only two-hundred
fifty-six possibilities for any given byte, the probability of success is simply
$\frac{1}{256}$. With such a reliable attack, the attacker can easily iterate
through the ciphertext blocks, shifting the bytes when a new byte has been
decrypted. This allows for full decryption of entire blocks given enough time.
Being able to make several hundred or thousand attacks in a few seconds means 
that an adversary can reliably decrypt one byte of a message at a time.

While this is fairly devastating, it should again be pointed out that an attacker
must already have a fairly compromised system in order to execute this method.
The attacker must have two requirements in place: the attacker must control
some part of the client side connection, and the attacker must have visibility
of the resulting ciphertext. These two prerequisites require a fairly
compromised system to begin with.\cite{US-CERT}

\section*{TLS 1.0 Vulnerability}
It was discovered a few months after the initial POODLE announcement was made
that certain implementations of TLS were vulnerable to the attack. TLS, unlike
SSLv3, is very strict in the formatting of its padding. Certain implementations,
however, did not check the padding structure after decryption. This leads to the
same vulnerability to padding oracle attacks that SSLv3 had, minus the need to
downgrade the security protocol being used.\cite{POODLEBites}

\section*{The Fix}
Ideally, it would be possible to patch SSLv3 to prevent the attack from occurring.
This could be done if SSLv3 altered its padding technique to be able to verify
its padding and padding byte. There has not, however, been such a patch released.
There are patches that make it much harder to execute the downgrade attack against
TLS capable machines. The patch only works, however, if both the client and the
server are up to date.\cite{disablessl3} The real danger lies in the fact
that SSLv3 is a deprecated protocol that should no longer be used. It is
extremely hard, if not impossible, to ensure that every server or client is
patched and up to date with the latest SSLv3 versions. More importantly, many
websites and modern browsers have completely disabled SSLv3 to prevent POODLE
from being an issue in the first place. Modern browsers enforce the use of TLS
1.2, and will fail to connect to older servers that do not support modern
encryption standards. The simple answer is to disable SSLv3 entirely.

POODLE is a fascinating example of a real world demonstration of a padding oracle
attack. While we learned about them this term in class, I did not expect to see
an attack as viable as this one out in the wild. It is, however, reassuring to see that
companies such as Google and Mozilla have taken such a hard line in protecting
their customers. I'm looking forward to reading more about attacks such as these
in the future. 

\bibliography{references}
\bibliographystyle{IEEEtran}
\end{document}
