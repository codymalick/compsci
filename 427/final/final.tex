\documentclass[10pt]{article}
%%%%%%%%%%%%%%%%%%%%%%%
% Mike's stuff

%% Sample LaTeX for CS427/519
%% Mike Rosulek
%% last update 2017-01-23

\usepackage{xspace,graphicx,amsmath,amssymb,xcolor}
\usepackage[margin=1in]{geometry}

%% operators

\newcommand{\pct}{\mathbin{\%}}
% makes ":=" aligned better
\usepackage{mathtools}
\mathtoolsset{centercolon}

% indistinguishability operator
% http://tex.stackexchange.com/questions/22168/triple-approx-and-triple-approx-with-a-straight-middle-line
\newcommand{\indist}{  \mathrel{\vcenter{\offinterlineskip
\hbox{$\sim$}\vskip-.35ex\hbox{$\sim$}\vskip-.35ex\hbox{$\sim$}}}}
\renewcommand{\cong}{\indist}

\newcommand{\K}{\mathcal{K}}
\newcommand{\M}{\mathcal{M}}
\newcommand{\C}{\mathcal{C}}
\newcommand{\Z}{\mathbb{Z}}

\newcommand{\Enc}{\text{\sf Enc}}
\newcommand{\Dec}{\text{\sf Dec}}
\newcommand{\KeyGen}{\text{\sf KeyGen}}

% fancy script L
\usepackage[mathscr]{euscript}
\renewcommand{\L}{\ensuremath{\mathscr{L}}\xspace}
\newcommand{\lib}[1]{\ensuremath{\L_{\textsf{#1}}}\xspace}

\newcommand{\myterm}[1]{\ensuremath{\text{#1}}\xspace}
\newcommand{\bias}{\myterm{bias}}
\newcommand{\link}{\diamond}
\newcommand{\subname}[1]{\ensuremath{\textsc{#1}}\xspace}

%% colors
\definecolor{highlightcolor}{HTML}{F5F5A4}
\definecolor{highlighttextcolor}{HTML}{000000}
\definecolor{bitcolor}{HTML}{a91616}

%%% boxes for writing libraries/constructions
\usepackage{varwidth}

\newcommand{\codebox}[1]{%
	\begin{varwidth}{\linewidth}%
		\begin{tabbing}%
			~~~\=\quad\=\quad\=\quad\=\kill % initialize tabstops
			#1
		\end{tabbing}%
	\end{varwidth}%
}
\newcommand{\titlecodebox}[2]{%
	\fboxsep=0pt%
	\fcolorbox{black}{black!10}{%
		\begin{varwidth}{\linewidth}%
			\centering%
			\fboxsep=3pt%
			\colorbox{black!10}{#1} \\
			\colorbox{white}{\codebox{#2}}%
		\end{varwidth}%
	}
}
\newcommand{\fcodebox}[1]{%
	\framebox{\codebox{#1}}%
}
\newcommand{\hlcodebox}[1]{%
	\fcolorbox{black}{highlightcolor}{\codebox{#1}}%
}
\newcommand{\hltitlecodebox}[2]{%
	\fboxsep=0pt%
	\fcolorbox{black}{black!15!highlightcolor}{%
		\begin{varwidth}{\linewidth}%
			\centering%
			\fboxsep=3pt%
			\colorbox{black!15!highlightcolor}{\color{highlighttextcolor}#1} \\
			\colorbox{highlightcolor}{\color{highlighttextcolor}\codebox{#2}}%
		\end{varwidth}%
	}
}


%% highlighting
\newcommand{\basehighlight}[1]{\colorbox{highlightcolor}{\color{highlighttextcolor}#1}}
\newcommand{\mathhighlight}[1]{\basehighlight{$#1$}}
\newcommand{\highlight}[1]{\raisebox{0pt}[-\fboxsep][-\fboxsep]{\basehighlight{#1}}}
\newcommand{\highlightline}[1]{%\raisebox{0pt}[-\fboxsep][-\fboxsep]{
	\hspace*{-\fboxsep}\basehighlight{#1}%
	%}
}

%% bits
\newcommand{\bit}[1]{\textcolor{bitcolor}{\texttt{\upshape #1}}}
\newcommand{\bits}{\{\bit0,\bit1\}}



%%%%%%%%%%%%%%%%%%%%%%%

\begin{document}
\title{CS 427, Final Project\\ POODLE\\ Padding Oracle on Downgraded Legacy
Encryption}
\author{Cody Malick\\
\texttt{malickc@oregonstate.edu}}
\date{\today}
\maketitle
\vspace{4cm}

\begin{abstract}
	This paper outlines the POODLE exploit. It gives details on how it
	allows an adversary to repeatedly query a server using SSL version 3.0,
	and eventually decrypt a block one byte at a time. The exploit was
	first published by the
	Google Security Team on October 14, 2014 by Bodo Moller, Thai Duong,
	and Kryzsztof Kotowicz. While SSL version 3.0 is quite old, it is still
	in use today for browser backward compatibility. Later, an updated
	version of the POODLE exploit was published showing successful attacks
	against TLS.

\end{abstract}

\clearpage

\section*{Introduction}
POODLE is an attack on Secure Socket Layer version 3.0 that was made public
in 2014. It was announced on Google's Security Blog, co-published by Bodo Moller,
Thai Duong, and Kryzsztof Kotowicz. POODLE allows an attacker targeting SSL 3.0
to decrypt one byte of an encrypted SSL block one out of two-hundred fifty-six 
attempts. While this is an average, being able to decrypt a byte of an encrypted
block that quickly is quite a security flaw.\cite{POODLE}

It was later found that the vulnerability extended to TLS version 1.0.

\section*{The Attack}
\subsection*{Context}
To completely understand the attack, the context should first be made clear. 
SSL version 3.0 is quite dated, and should for all intents and purposes be 
completely retired. Modern browsers, however, use SSLv3 as a fall back for
compatibility purposes. The browser will first try to connect to the web server
using TLS version 1.2, the de facto standard for web communication. If that fails,
it is default behavior in browsers to fall back to earlier standards, such as SSLv3.\cite{POODLE}

While browsers continue to support SSLv3, this attack will continue to be an
issue. A savvy attacker with the proper resources could catch web requests with
a man in the middle attack, and prevent all non-SSLv3 requests from going through.
This would force the use of SSLv3 if it is supported by the web server. 

\subsection*{Padding}
The primary point of vulnerability that the attack exploits is in the padding
of the encrypted message. Padding in SSLv3 is accomplished by measuring the 
difference between the last block size, and the required size of the block. In
this case, the block size must be a multiple of sixteen. The SSLv3 implementation
then measures the difference $d$, and fills the space $d-1$ with random byte
values. The last byte is then filled with the size of the padding. 
\subsection*{AES CBC Mode}
\subsection*{Exploit}

\section*{Exploit Publication by Google}
\section*{The Fix}

\section*{Bibliography}
\bibliography{references}
\bibliographystyle{IEEEtran}
\end{document}
