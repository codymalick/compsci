\documentclass[10pt,letterpaper]{article}

\usepackage[margin=0.75in]{geometry}
\usepackage{tikz}
\usepackage{amsfonts}
\usepackage{listings}


\lstset{frame=tb,
	language=C,
	columns=flexible,
	numberstyle=\tiny\color{gray},
	keywordstyle=\color{blue},
	commentstyle=\color{dkgreen},
	stringstyle=\color{mauve},
	breaklines=true,
	breakatwhitespace=true,
	tabsize=4
}

\begin{document}

  \title{CS 427, Assignment 1}
  \author{Cody Malick\\
  \texttt{malickc@oregonstate.edu}}
  \date{\today}
  \maketitle

\section{}
	We can find the value of $k$ by xoring $m$ and $c$\\\\
	$m \oplus c:$\\\\
	$110101100010110\\
	100111011011111\\
	\\
	010010111001001$\\\\
	\noindent Then, $k=010010111001001$. We can use this info to then xor
	the value of $m'$ with the value of $k$:\\
	
	\noindent $m' \oplus k:\\\\
	010010111001001\\
	001000000101111\\
	\\
	011010111100110$\\\\
	
	\noindent Then our encrypted $m'$, lets call it $c' = 011010111100110$

\section{}
	We can show that these two libraries are not interchangable by
	inspecting the distribution of each library:\\

	$\mathcal{L}_1$ has a probability distribution of
	$\{0,1\}^\lambda - 0^\lambda$\\\\

	$\mathcal{L}_2$, on the other hand, has a probability distribution of
	$\{0,1\}^\lambda$\\\\

	This makes it quite obvious that there is a difference between the two
	distributions. Specifically, that $\mathcal{L}_2$ can produce the key
	consisting of only 0's, while $\mathcal{L}_1$ cannot. While a key
	consisting of only zeroes isn't ideal, it is a difference in behavior
	which Eve can exploit. In an example case:\\

	If calling program $F(string\ m)$ which calls $VIEW(m)$, the adversary
	Eve can't choose which function, but she can choose the message $m$.
	Using function $F$:\\
	\clearpage

	\begin{lstlisting}
	F(string m) {
		result = VIEW(m);
		if m == result {
			return true;
		}
		
		return false;
	}
	\end{lstlisting}
	\noindent When $F$ calls $VIEW(m)$ with $\mathcal{L}_1$:$P(VIEW(m)) $ returning false
	$= 1$\\
	When $F$ calls $VIEW(m)$ with $\mathcal{L}_2$:$P(VIEW(m))$ returning false $=
	\frac{1}{2^\lambda}$\\
	
	In the case of $\mathcal{L}_2$, it simply
	returns a random ciphertext $c \leftarrow \{0,1\}^\lambda$, while
	$\mathcal{L}_1$ returns an OTP encrypted ciphertext with a random key
	without the possibility of an all zero key. Because the probabilities
	differ, there is a case where Eve can find a difference in behavior,
	specifically that $\mathcal{L}_1$ cannot produce an all zero key. 

\section{}
	We can show these two functions, $\mathcal{L}_1, \mathcal{L}_2$ are 
	interchangable by slowly changing bits of $\mathcal{L}_1$ to look like
	$\mathcal{L}_2$. Our end goal is to show that $\mathcal{L}_1$ produces
	values are in the same range that $\mathcal{L}_2$ does.\\

	First, it is important to note the probability distribution of each
	function. $\mathcal{L}_1$ takes in a base $n$ integer, and outputs
	a random value $c \in \mathbb{Z}_n$. Given that integer input,
	$\mathcal{L}_1$ then generates a random key, adds the key to the integer
	input, and uses a modulus function to maintain the domain of
	$\mathbb{Z}_n$. Then the probability of a single output occuring in the
	function is $\frac{1}{n}$.\\

	$\mathcal{L}_2$ has identical input behavior, but different
	functionality inside the function. It simply returns a random ciphertext
	in the domain specified, $\mathbb{Z}_n$. The probability of a single
	result then, is $\frac{1}{n}$.

	We can show these functions are equivilant by slowly changing one to
	look like the other:

	\begin{enumerate}
		\item First, we can add a line to $\mathcal{L}_2$, setting
			$c = c \bmod n$. This does not change the behavior of
			the library, as $n$ is the size of the input domain,
			$\mathbb{Z}_n$. Our function now looks like this:
			\begin{lstlisting}[frame=none]
				VIEW(M):
					c <-- Z_n
					c = c % n
					return c
			\end{lstlisting}

		\item Change initial $c$ assignment to $k$ to store initial
			key value in a seperate variable. This does not alter
			the functionality of the library as we are simply
			storing the random value in a different variable. 
			Our function now looks like this:
			\begin{lstlisting}[frame=none]
				VIEW(M):
					k <-- Z_n
					c := k % n
					return c
			\end{lstlisting}
		\item Lastly, we can then add $x$ to $k$ without consequence
			to the library as the modulus function maintains the
			domain of output. The randomness of output is also not
			changed as it simply offsets all possibilities by a 
			constant amount. Offsetting a uniform distribution by
			a constant amount does not change the resulting value.
			Now our function looks like this:
			\begin{lstlisting}[frame=none]
				VIEW(M):
					k <-- Z_n
					c := (x + k) % n
					return c
			\end{lstlisting}
	\end{enumerate}

	Our functions are now identical, then $\mathcal{L}_1 \equiv 
	\mathcal{L}_2$, and therefore interchangable.
\end{document}
