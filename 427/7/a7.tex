\documentclass[10pt]{article}
%%%%%%%%%%%%%%%%%%%%%%%
% Mike's stuff

%% Sample LaTeX for CS427/519
%% Mike Rosulek
%% last update 2017-01-23

\usepackage{xspace,graphicx,amsmath,amssymb,xcolor}
\usepackage[margin=1in]{geometry}

%% operators

\newcommand{\pct}{\mathbin{\%}}
% makes ":=" aligned better
\usepackage{mathtools}
\mathtoolsset{centercolon}

% indistinguishability operator
% http://tex.stackexchange.com/questions/22168/triple-approx-and-triple-approx-with-a-straight-middle-line
\newcommand{\indist}{  \mathrel{\vcenter{\offinterlineskip
\hbox{$\sim$}\vskip-.35ex\hbox{$\sim$}\vskip-.35ex\hbox{$\sim$}}}}
\renewcommand{\cong}{\indist}

\newcommand{\K}{\mathcal{K}}
\newcommand{\M}{\mathcal{M}}
\newcommand{\C}{\mathcal{C}}
\newcommand{\Z}{\mathbb{Z}}

\newcommand{\Enc}{\text{\sf Enc}}
\newcommand{\Dec}{\text{\sf Dec}}
\newcommand{\KeyGen}{\text{\sf KeyGen}}

% fancy script L
\usepackage[mathscr]{euscript}
\renewcommand{\L}{\ensuremath{\mathscr{L}}\xspace}
\newcommand{\lib}[1]{\ensuremath{\L_{\textsf{#1}}}\xspace}

\newcommand{\myterm}[1]{\ensuremath{\text{#1}}\xspace}
\newcommand{\bias}{\myterm{bias}}
\newcommand{\link}{\diamond}
\newcommand{\subname}[1]{\ensuremath{\textsc{#1}}\xspace}

%% colors
\definecolor{highlightcolor}{HTML}{F5F5A4}
\definecolor{highlighttextcolor}{HTML}{000000}
\definecolor{bitcolor}{HTML}{a91616}

%%% boxes for writing libraries/constructions
\usepackage{varwidth}

\newcommand{\codebox}[1]{%
	\begin{varwidth}{\linewidth}%
		\begin{tabbing}%
			~~~\=\quad\=\quad\=\quad\=\kill % initialize tabstops
			#1
		\end{tabbing}%
	\end{varwidth}%
}
\newcommand{\titlecodebox}[2]{%
	\fboxsep=0pt%
	\fcolorbox{black}{black!10}{%
		\begin{varwidth}{\linewidth}%
			\centering%
			\fboxsep=3pt%
			\colorbox{black!10}{#1} \\
			\colorbox{white}{\codebox{#2}}%
		\end{varwidth}%
	}
}
\newcommand{\fcodebox}[1]{%
	\framebox{\codebox{#1}}%
}
\newcommand{\hlcodebox}[1]{%
	\fcolorbox{black}{highlightcolor}{\codebox{#1}}%
}
\newcommand{\hltitlecodebox}[2]{%
	\fboxsep=0pt%
	\fcolorbox{black}{black!15!highlightcolor}{%
		\begin{varwidth}{\linewidth}%
			\centering%
			\fboxsep=3pt%
			\colorbox{black!15!highlightcolor}{\color{highlighttextcolor}#1} \\
			\colorbox{highlightcolor}{\color{highlighttextcolor}\codebox{#2}}%
		\end{varwidth}%
	}
}


%% highlighting
\newcommand{\basehighlight}[1]{\colorbox{highlightcolor}{\color{highlighttextcolor}#1}}
\newcommand{\mathhighlight}[1]{\basehighlight{$#1$}}
\newcommand{\highlight}[1]{\raisebox{0pt}[-\fboxsep][-\fboxsep]{\basehighlight{#1}}}
\newcommand{\highlightline}[1]{%\raisebox{0pt}[-\fboxsep][-\fboxsep]{
	\hspace*{-\fboxsep}\basehighlight{#1}%
	%}
}

%% bits
\newcommand{\bit}[1]{\textcolor{bitcolor}{\texttt{\upshape #1}}}
\newcommand{\bits}{\{\bit0,\bit1\}}



%%%%%%%%%%%%%%%%%%%%%%%

\begin{document}
\title{CS 427, Assignment 7}
\author{Cody Malick\\
\texttt{malickc@oregonstate.edu}}
\date{\today}
\maketitle

\section{}
To get the value of $m$, we ultimately have to derive the value of $d$, our
decryption exponent. To do so, we start by calculating the RSA Modulus:\\

\noindent $N\ =\ p*q$\\
$N\ =\ 911797486545590530685490581969705143802653291668718088807630693862734987\\
5532657604237614264588159197059815174375567981463992373512122956300201579359580\\
5215238577716681027825152983443943441550557354571582134685037600218456613417870\\
18714549993329931085779672821202518040360247980809810895818474245240280337281$\\

\noindent Now we need the value of $\phi(N)$:\\

\noindent $\phi(N)\ =\ (p-1)*(q-1)$\\
$\phi(N)\ =\ 911797486545590530685490581969705143802653291668718088807630693862\\
7349875532657604237614264588159197059815174375567981463992373512122956300201579\\
3595805077546189468415157627748782583976339715090578658910915643377622180520960\\
9946226945875923321513495389884349680225795943502744149049896011236139948560364\\
6464$\\

\noindent With those two numbers, we can calculate the value of $d$:\\

\noindent $d\ =\ \frac{1}{e\ mod\ \phi(N)}$\\
$d\ =\ 876087402216128760554205050038776743914820917321379260264250901674899413\\
6282031327309248170674784815973921629687229809866185883687888375766251125859858\\
0335605738183751378608385253239956378629643166962086728346639595202171576099271\\
93631392683611409041866251714943161694705196079552228765852086148853008203391$\\

\noindent Now that we know $d$, simply raise our ciphertext to the power of $d$:\\

\noindent $m\ =\ (c\ mod\ N)^d$\\
$m\ =\ 1223334444555556666667777777888888889999999990000000000$\\

And our message is ``1223334444555556666667777777888888889999999990000000000''. I
confirmed this by re-encrypting it with the encryption exponent $e$.

\section{}
Given $\phi(N)$ and $n$, we can quickly find the two factors using algebra, and the
quadratic equation:

\noindent $\phi(N)\ =\ (p-1)(q-1)$\\
$N\ =\ p*q$\\

\noindent $\phi(n)\ =\ (p-1)*(q-1)$\\
$\frac{\phi(N)}{p-1}\ =\ (q-1)$\\
$\frac{\phi(N)}{p-1}+1\ =\ q$\\

\noindent And conversely:\\

\noindent$\frac{\phi(n)}{q-1}+1\ =\ p$\\

\noindent We also have the other formula:\\

\noindent $p\ =\ \frac{N}{q}$\\
$q\ =\ \frac{N}{p}$\\

Plugging these in to create two quadratic equations:

\noindent $0\ =\ q^2+(Nq+q-\phi(N)q)+N$\\
\noindent $0\ =\ p^2+(Np+p-\phi(N)p)+N$\\

Plug these into the quadratic formula:

\noindent $q\ =\ \frac{(1+N-\phi(N)) + \sqrt{(1+N-\phi(N))^2-4*N}}{2}$\\
\noindent $p\ =\ \frac{(1+N-\phi(N)) - \sqrt{(1+N-\phi(N))^2-4*N}}{2}$\\

Using these formula, we can calculate the requested values:\\

\noindent $p\ =\ 13071703506582537746030691218059776645988993761791975581514478\\
3550384410764373225534882490634747346714029248654758565923709885799257583872347\\
06493900130209$\\

\noindent $q\ =\ 69753531824404927370972886793693353755768382947514632265151944\\
8755124165887426702302511051321397209426399534784224332849550739386177318878139\\
260776560609$\\

$p*q\ =\ N$

\section{}
If $x$ and $y$ are known, and we know that $x^2 \equiv_N y^2$, then:\\

\noindent $x^2\ =\ y^2\ mod\ N$\\

Because of this relationship, we know also that:

\noindent $x^2-y^2\ =\ 0\ mod\ N$\\
$(x-y)(x+y)\ =\ 0\ mod\ N$\\

Because we know that the left side of the equation is equal to 0 mod n, the
left side is then equal to some multiple of n. With that knowledge, we know that
$(x-y)$ and $(x+y)$ contain factors of N. We can then solve for the greatest
common divisor of $((x+y),N)$ and $((x-y),N)$ to quickly find the factors. 

\end{document}
