\documentclass[10pt,letterpaper]{article}

\usepackage[margin=0.75in]{geometry}
\usepackage{tikz}
\usepackage{amsfonts}
\usepackage{xspace}
\usepackage{graphicx}
\usepackage{mathtools}
\usepackage{amssymb}
\usepackage{listings}
\usepackage{amsmath}
\usepackage{color}
\usepackage{empheq}
\usepackage{bm}
\definecolor{myblue}{rgb}{.8, .8, 1}
\newlength\mytemplen
\newsavebox\mytempbox

\makeatletter
\newcommand\mybluebox{%
\@ifnextchar[%]
	{\@mybluebox}%
{\@mybluebox[0pt]}}

\def\@mybluebox[#1]{%
	\@ifnextchar[%]
		{\@@mybluebox[#1]}%
	{\@@mybluebox[#1][0pt]}}

\def\@@mybluebox[#1][#2]#3{
	\sbox\mytempbox{#3}%
	\mytemplen\ht\mytempbox
	\advance\mytemplen #1\relax
	\ht\mytempbox\mytemplen
	\mytemplen\dp\mytempbox
	\advance\mytemplen #2\relax
	\dp\mytempbox\mytemplen
\colorbox{myblue}{\hspace{1em}\usebox{\mytempbox}\hspace{3em}}}

\makeatother

\begin{document}
\title{CS 427, Assignment 3}
\author{Cody Malick\\
\texttt{malickc@oregonstate.edu}}
\date{\today}
\maketitle

\section{}
	\subsection*{a}
	This PRG is insecure. We can show this by having two QUERY functions,
	$QUERY_1$ and $QUERY_2$. $QUERY_2$ returns a uniform random distribution,
	while $QUERY_1$ uses a psuedorandom function $H(s)$:\\

	\begin{empheq}[box={\mybluebox[3pt]}]{equation*}
		\begin{split}
			QUERY_1():\\
			& \quad s\ \leftarrow\ \{0,1\}^{\lambda}\\
			& \quad return\ H(s)\\
		\end{split}
	\end{empheq}

	\begin{empheq}[box={\mybluebox[3pt]}]{equation*}
		\begin{split}
			QUERY_2():\\
			& \quad z\ \leftarrow\ \{0,1\}^{2\lambda}\\
			& \quad return\ z\\
		\end{split}
	\end{empheq}

	I claim that an adversary can tell the difference between these two
	functions by only looking at the result of calling either function. 
	The adversary can tell which library he is calling, by simply looking
	at the second half of the output of $QUERY_1$. The calling function
	$\mathcal{F}$:\\

	\begin{empheq}[box={\mybluebox[3pt]}]{equation*}
		\begin{split}
			\mathcal{F}():\\
			& \quad x\ :=\ QUERY()\\
			& \quad x_1\ :=\ first\ half\ of\ x\\
			& \quad x_2\ :=\ second\ half\ of\ x\\
			& \quad \\
			& \quad y\ :=\ QUERY()\\
			& \quad y_1\ :=\ first\ half\ of\ y\\
			& \quad y_2\ :=\ second\ half\ of\ y\\
			& \quad \\
			& \quad if\ (\ x_2\ ==\ y_2\ )\ \{\\
			& \qquad return\ true \\
			& \quad \} \\
			& \quad return\ false\\
		\end{split}
	\end{empheq}

	Given that $\mathcal{F}$ calls $QUERY_1$ twice, then $\mathcal{F}$ will
	return true with a probability of 1. While if $\mathcal{F}$ calls 
	$QUERY_2$ twice, then the probability of $x_2 == y_2$ is quite small,
	$\frac{1}{2^\lambda}*\frac{1}{2^\lambda}$. Given that information, the
	PRG is not secure.

	\subsection*{b}

	I claim this PRG is secure due to OTP. To show that this PRG is secure,
	we must show that, given a calling function, we cannot differentiate 
	between two queries, $QUERY_1,QUERY_2$:

	\begin{empheq}[box={\mybluebox[3pt]}]{equation*}
		\begin{split}
			QUERY_1():\\
			& \quad s\ \leftarrow\ \{0,1\}^{\lambda}\\
			& \quad return\ H(s)\\
		\end{split}
	\end{empheq}

	\begin{empheq}[box={\mybluebox[3pt]}]{equation*}
		\begin{split}
			QUERY_2():\\
			& \quad z\ \leftarrow\ \{0,1\}^{2\lambda}\\
			& \quad return\ z\\
		\end{split}
	\end{empheq}

	Inside of $H(s)$, getting two random strings and xoring them together
	looks a lot like OTP. We can try and convert $H(s)$ into a different form.
	With the original $H(s)$:

	\begin{empheq}[box={\mybluebox[3pt]}]{equation*}
		\begin{split}
			H(s):\\
			& \quad x\ :=\ G(s)\\
			& \quad y\ :=\ G(0^\lambda) \\
			& \quad return\ x \oplus y\\
		\end{split}
	\end{empheq}

	We can abstract out the xor into a function $\mathcal{F}$:

	\begin{empheq}[box={\mybluebox[3pt]}]{equation*}
		\begin{split}
			\mathcal{F}(k,m):\\
			& \quad return\ k \oplus m
		\end{split}
	\end{empheq}

	\begin{empheq}[box={\mybluebox[3pt]}]{equation*}
		\begin{split}
			H(s):\\
			& \quad x\ :=\ G(s)\\
			& \quad y\ :=\ G(0^\lambda) \\
			& \quad return\ \bm{\mathcal{F}(x,y)} \\
		\end{split}
	\end{empheq}

	I claim this does not change the calling function, as we are performing
	the same operations, just with a linking function rather than in $H(s)$.
	$\mathcal{F}$ is identical to OTP, which we have proven has a uniformly
	distributed output. With that in mind, both queries, $QUERY_1,QUERY_2$
	both return a uniformly distributed output, and we can differentiate between
	them. They are, then, indistinguishable. $H()$ is a secure PRG.


	\subsection*{c}
	This problem is very similar to the first problem, but the primary difference
	is that the function will \texttt{always} return the seed as the first
	half of the return value. I claim that this function is secure, as the
	seed is randomly generated from a uniform distribution:

	\begin{empheq}[box={\mybluebox[3pt]}]{equation*}
		\begin{split}
			H(s):\\
			& \quad x\ :=\ G(s)\\
			& \quad return\ s||x\\
		\end{split}
	\end{empheq}

	We can first abstract the call to $G(s)$ as a uniformly random distribution,
	as we know that $G()$ is secure:

	\begin{empheq}[box={\mybluebox[3pt]}]{equation*}
		\begin{split}
			H(s):\\
			& \quad x\ \leftarrow \{0,1\}^{2\lambda}\\
			& \quad return\ s||x \\
		\end{split}
	\end{empheq}

	It is clear now that we have two values, both of which have a uniform
	random distribution. The resulting probability distribution is then
	$\frac{1}{2^{2\lambda}} * \frac{1}{2^{\lambda}} = \frac{1}{2^{3\lambda}}$.
	With that output, then $H()$ is a secure PRG

\section{}
This is a relatively simplistic proof. Our end goal is simply to show that we
can freely exchange $G_1$ and $G_2$. Starting with $\mathcal{L}^{G_1}_{which-prg}$:\\

\begin{empheq}[box={\mybluebox[3pt]}]{equation*}
	\begin{split}
		QUERY():\\
		& \quad s\ \leftarrow\ \{0,1\}^{\lambda}\\
		& \quad return\ G_1(s)\\
	\end{split}
\end{empheq}

My first step would be abstract away the call to $G_1$ into a seperate function:\\

\begin{empheq}[box={\mybluebox[3pt]}]{equation*}
	\begin{split}
		QUERY():\\
		& \quad s\ \leftarrow\ \{0,1\}^{\lambda}\\
		& \quad \bm{z\ :=\ \mathcal{F}(s)} \\
		& \quad return\ \bm{z}
	\end{split}
\end{empheq}

Where $\mathcal{F}$:\\

\begin{empheq}[box={\mybluebox[3pt]}]{equation*}
	\begin{split}
		\mathcal{F}(s):\\
		& \quad z\ := G_1(s)\\
		& \quad return\ z
	\end{split}
\end{empheq}

Because we assume that $G_1$ is a secure PRG, then we can simply set the output
of the function to a uniformly random distribution:

\begin{empheq}[box={\mybluebox[3pt]}]{equation*}
	\begin{split}
		\mathcal{F}(s):\\
		& \quad \bm{z\ \leftarrow\ \{0,1\}^{2\lambda}}\\
		& \quad return\ z
	\end{split}
\end{empheq}

We can take advantage of the fact that the output is uniformly random, to substitute
$G_2$ in place of the uniform distribution, as we assume $G_2$ is a secure PRG:
input and output ranges:

\begin{empheq}[box={\mybluebox[3pt]}]{equation*}
	\begin{split}
		\mathcal{F}(s):\\
		& \quad z\ := \bm{G_2(s)}\\
		& \quad return\ z
	\end{split}
\end{empheq}

With that changed, we can substitute the resulting function back into $QUERY()$:\\

\begin{empheq}[box={\mybluebox[3pt]}]{equation*}
	\begin{split}
		QUERY():\\
		& \quad s\ \leftarrow\ \{0,1\}^{\lambda}\\
		& \quad \bm{z\ :=\ \bm{G_2(s)}} \\
		& \quad return\ z
	\end{split}
\end{empheq}

With that in place, we've shown that $\mathcal{L}^{G_1}_{which-prg}$ is indistinguishable
from $\mathcal{L}^{G_2}_{which-prg}$

\end{document}
