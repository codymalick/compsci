\documentclass[10pt,letterpaper]{article}

\usepackage[margin=0.75in]{geometry}
\usepackage{tikz}
\usepackage{amsfonts}
\usepackage{listings}
\usepackage{amsmath}
\usepackage{color}
\usepackage{empheq}
\definecolor{myblue}{rgb}{.8, .8, 1}
\newlength\mytemplen
\newsavebox\mytempbox

\makeatletter
\newcommand\mybluebox{%
\@ifnextchar[%]
	{\@mybluebox}%
{\@mybluebox[0pt]}}

\def\@mybluebox[#1]{%
	\@ifnextchar[%]
		{\@@mybluebox[#1]}%
	{\@@mybluebox[#1][0pt]}}

	\def\@@mybluebox[#1][#2]#3{
		\sbox\mytempbox{#3}%
		\mytemplen\ht\mytempbox
		\advance\mytemplen #1\relax
		\ht\mytempbox\mytemplen
		\mytemplen\dp\mytempbox
		\advance\mytemplen #2\relax
		\dp\mytempbox\mytemplen
	\colorbox{myblue}{\hspace{1em}\usebox{\mytempbox}\hspace{3em}}}

	\makeatother
	\begin{document}
	\title{CS 427, Assignment 2}
	\author{Cody Malick\\
	\texttt{malickc@oregonstate.edu}}
	\date{\today}
	\maketitle

	\section{}
	The one-time secrecy property of an encryption scheme is defined as
	having the ability to be changed with another library, and not chaning
	the resulting probability distribution. Given that information, if we
	can show that $\Sigma \equiv \Sigma^2$, then we can say that is 
	satisfies one-time secrecy.

	Using the notation provided, $\Sigma$ is defined by its three function,
	keygen, encryption, and decryption. Then, to show that they are equivilant,
	we have to show both encryption functions are equal:

	\begin{empheq}[box={\mybluebox[3pt]}]{equation*}
		\begin{split}
			Enc((k_1,k_2),m):\\
			& \quad c_1 \leftarrow \Sigma.Enc(k_1,m)\\
	  & \quad c_2 \leftarrow \Sigma.Enc(k_2,m)\\
	  & \quad return(c_1,c_2)\\
		\end{split}
	\end{empheq}
	First, we can replace $\Sigma.Enc()$ with an equivilent function,
	such as OTP:\\

	\begin{empheq}[box={\mybluebox[3pt]}]{equation*}
		\begin{split}
			Enc((k_1,k_2),m): \\
			& \quad c_1 \leftarrow OTP.Enc(k_1,m)\\
	  & \quad c_2 \leftarrow OTP.Enc(k_2,m)\\
	  & \quad return (c_1,c_2)
		\end{split}
	\end{empheq}
	We can do this because we have shown in the past that the output of
	OTP is uniformly random, and has the one-time secrecy property. Assuming
	that $\Sigma$ has this property, then the output of the calling function
	is unchanged.\\

	Next, we can simply replace $c_1$ and $c_2$ with random values $z_1$
	and $z_2$ as the output of OTP is uniformly random:\\

	\begin{empheq}[box={\mybluebox[3pt]}]{equation*}
		\begin{split}
			Enc((k_1,k_2),m): \\
			& \quad z_1 \leftarrow OTP.Enc(k_1,m)\\
	  & \quad z_2 \leftarrow OTP.Enc(k_2,m)\\
	  & \quad return (z_1,z_2)
		\end{split}
	\end{empheq}

	Because the output of the function is uniformly random, regardless of
	input, we can substitute the message $m$ for any value, as the output
	for $\mathcal{L}^{\Sigma^2}_{ots-L}$ is completely random, and the
	output for $\mathcal{L}^{\Sigma^2}_{ots-R}$ is completely random. 
	Because of the random output, the two messages are interchangable,
	and $\Sigma^2$ has one-time secrecy

	\section{}
	The definition of one-time secrecy is that a calling program, given
	appropriate inputs, cannot differentiate between two libraries. More
	specifically, the output distributions of both libraries should be 
	identitical. We can show that a library does not have one-time secrecy
	by showing a situation where the resulting probability distribution
	of one input is not the same as the rest.

	Suppose we have a calling function $A$:\\

	\begin{empheq}[box={\mybluebox[3pt]}]{equation*}
		\begin{split}
			func\ bool A(): \\
			&\quad mes_1\ :=\ 1 \\
	  &\quad k\ :=\ KeyGen() \\
	  &\quad c\ :=\ Enc(k,mes_1) \\
	  &\quad if(c\ ==\ k) \{ \\
	  &\qquad return\ true \\
	  &\quad \} \\
	  &\quad return\ false \\
		\end{split}
	\end{empheq}
	Picking the value 1 for the message, function $A$ will always return 
	true. The value of the key, as we encrypt by multiplying the key by the
	message value. $1*k \bmod 10$ will always return $k$. This means that,
	from the ciphertext, we can deduce the value of the message as we know
	the value of the key. Being able to get information about the original
	message proves that one-time secrecy does not apply here.

	\section{}
		To find the secret, we need to solve for the polynomial expression
		that defines the secret share. The line is defined by a
		polynomial of type $t-1$, where $t=$ the threshold. In this
		case, we need to find a poly function of $t-1=1$, so a simple
		linear equation. The secret we want to find is $f(0)$, the
		$y$ intercept. Our equation then:\\

		$y=mx+b$\\

	\noindent First, we can solve for the slope:\\

		$\frac{3-6}{7-4}=-1$\\

	\noindent Then:\\

		$y=-x+b$\\

	\noindent Next, plug in a point and solve for the intercept:\\

		$3=-7+b$\\

		$10=b$\\

	\noindent Then we have our equation, and incidently our secret, 10

	\noindent We can find the rest of our shares by plugging in the missing
	shares, $1 \rightarrow 10$:\\

	Share 0: $0+10=10$\\

	Share 1: $-1+10=9$\\

	Share 2: $-2+10=8$\\

	Share 3: $-3+10=7$\\

	Share 5: $-5+10=5$\\

	Share 6: $-6+10=4$\\

	Share 8: $-8+10=2$\\

	Share 9: $-9+10=1$\\

	Share 10: $-10+10=0$\\

	Given two of these other shares, we could have used them to find the
	slope of our equation. 

\end{document}
