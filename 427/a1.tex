\documentclass[10pt,letterpaper]{article}

\usepackage[margin=0.75in]{geometry}
\usepackage{tikz}
\usepackage{amsfonts}
\begin{document}

  \title{CS 427, Assignment 1}
  \author{Cody Malick\\
  \texttt{malickc@oregonstate.edu}}
  \date{\today}
  \maketitle

\section{}
	We can find the value of $k$ by xoring $m$ and $c$\\\\
	$m \oplus c:$\\\\
	$110101100010110\\
	100111011011111\\
	\\
	010010111001001$\\\\
	\noindent Then, $k=010010111001001$. We can use this info to then xor
	the value of $m'$ with the value of $k$:\\
	
	$m' \oplus k:\\\\
	010010111001001\\
	001000000101111\\
	\\
	011010111100110$\\\\
	
	\noindent Then our encrypted $m'$, lets call it $c' = 011010111100110$

\section{}
	We can show that these two libraries are not interchangable by
	inspecting the distribution of each library:\\

	$\mathcal{L}_1$ has a probability distribution of
	$\{0,1\}^\lambda - 0^\lambda$\\\\

	$\mathcal{L}_2$, on the other hand, has a probability distribution of
	$\{0,1\}^\lambda$\\\\

	This makes it quite obvious that there is a difference between the two
	distributions. Specifically, that $\mathcal{L}_2$ can produce the key
	consisting of only 0's, while $\mathcal{L}_1$ cannot. While a key
	consisting of only zeroes isn't ideal, it is a difference in behavior
	which Eve can exploit. In an example case:\\

	If calling program $prog(m_1)$ which calls $\mathcal{L}_1$ or 
	$\mathcal{L}_2$, the adversary Eve can't choose the function, but she
	can choose the message $m_1$. In the case of $\mathcal{L}_2$, it simply
	returns a random ciphertext $c \leftarrow \{0,1\}^\lambda$, while
	$\mathcal{L}_2$ returns an OTP encrypted ciphertext with a random key
	without the possibility of an all zero key. The probabilities of 
	$\mathcal{L}_1$ and $\mathcal{L}_2$ are as follows (where $P(x)$ is the
	probability of x):\\

	\noindent$P(\mathcal{L}_1) = 1-P(0^\lambda)$\\
	$P(\mathcal{L}_2) = 1$\\

	Because the probabilities differ, there is a case where Eve can find a
	difference in behavior, specifically that $\mathcal{L}_1$ cannot
	produce an all zero key. 

\section{}
	We can show these two functions, $\mathcal{L}_1, \mathcal{L}_2$ are 
	interchangable by slowly changing bits of $\mathcal{L}_1$ to look like
	$\mathcal{L}_2$. Our end goal is to show that $\mathcal{L}_1$
	values in the same range that $\mathcal{L}_2$ does:

	\begin{enumerate}
		\item change $\mathcal{L}_2$ to include: $c = c \bmod n$\\
			This doesn't change the calling function as the modulus
			function is the size of $\mathbb{Z}$, leaving the
			library behavior unchanged
		\item 
		\item 
	\end{enumerate}

	

\section{}
\end{document}
