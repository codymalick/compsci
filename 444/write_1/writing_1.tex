\documentclass[10pt,letterpaper,onecolumn,draftclsnofoot]{IEEEtran}
\usepackage[margin=0.75in]{geometry}
\usepackage{listings}
\usepackage{color}
\usepackage{longtable}
\usepackage{tabu}
\usepackage{url}
\definecolor{dkgreen}{rgb}{0,0.6,0}
\definecolor{gray}{rgb}{0.5,0.5,0.5}
\definecolor{mauve}{rgb}{0.58,0,0.82}

\lstset{frame=tb,
  language=C,
  columns=flexible,
  numberstyle=\tiny\color{gray},
  keywordstyle=\color{blue},
  commentstyle=\color{dkgreen},
  stringstyle=\color{mauve},
  breaklines=true,
  breakatwhitespace=true,
  tabsize=4
}

\begin{document}
\begin{titlepage}
  \title{CS 444 - Spring 2016 - Writing Assignment 1}
  \author{Cody Malick\\
  \texttt{malickc@oregonstate.edu}}
  \date{April 21, 2016}
  \maketitle
  \vspace*{4cm}
  \begin{abstract}
      \noindent Understanding how an operating system handles processes, threads,
      and CPU scheduling are important topics for an aspiring kernel developer.
      In this report, we will cover how Linux, Windows, and FreeBSD handle these
      important structures in their respective kernels, and contrast them.
  \end{abstract}
\end{titlepage}

\tableofcontents
\clearpage
\section{Introduction}
This report contains descriptions on how each of the operating systems, Linux,
Windows, and FreeBSD handle processes, threads, and CPU scheduling. A lot can be
learned about the design decisions of the minds behind the operating system by
how they handle these core constructs. We will start by describing how Linux
handles these, as Linux is the focus of this class.

\section{Linux}
  \subsection{Processes}
The first thing that should be defined, is simply what a process is: "A
\textit{process} is a program (object code stored on some media) in the midst of
execution."\cite{robertlove2010} Although the Linux kernel refers to processes
as tasks, in this report (for uniformity) we will only refer to them as processes.

A process in Linux is ultimately born, somewhere up the line, from the
\texttt{init} process. The \texttt{init} is the absolute root of every process
tree on any Linux machine. It has a process ID of 1. All processes are spawned
from either the \texttt{fork()} or \texttt{exec()} function calls. \texttt{Fork()}
creates an exact copy of the calling process as a child of the process, while
\texttt{exec()} loads a new executable into the process space and executes it.
These are the only ways new processes are created in Linux.

Each process has a process descriptor that describes every aspect of
the process, the \texttt{task\_struct}. The \texttt{task\_struct} is contained
in \texttt{<linux/sched.h>}. \texttt{Task\_struct} describes the current state
of the process, the parent process ID, the process ID, exit code, priority, etc.
Everything you could possibly want to know about a process in Linux is in that
struct. I've included the \texttt{Task\_struct} as Appendix A as it is a very
large structure.

Processes, of course, must have states of execution. On Linux, processes have
five states, each represented by flags in the \texttt{task\_struct}:

\begin{description}
  \item \texttt{TASK\_RUNNING}: The process is either running or in a run-queue
  ready to be run (run-queues will be discussed in the CPU scheduling section).
  \item \texttt{TASK\_INTERRUPTIBLE}: the process is sleeping (blocked) waiting
  for a condition to un-block. Once the condition is met, it transitions to
  \texttt{TASK\_RUNNING}.
  \item \texttt{TASK\_UNINTERRRUPTIBLE}: The same state as \texttt{TASK\_INTERRUPTIBLE}
  with the difference being that it does not wake once an interrupt is received.
  \item \texttt{\_TASK\_TRACED}: This flag shows that the process is being traced
  by another process. An example is GDB.
  \item \texttt{\_TASK\_STOPPED}: Process execution has halted and it is not
  possible to start running. This state is a result of a halting interrupt from
  the CPU, or if a debugger sends it a signal.
\end{description}

  \subsection{Threads}
As with processes, we will define what a thread is: "Threads of execution, often
shortened to \textit{threads}, are the objects of activity within the process."
\cite{robertlove2010} In Linux, threads are treated fundamentally the same as
processes. The only difference is that they have a shared address space,
filesystem resources, file descriptors, and signal handlers.

Here is the \texttt{thread\_info} struct:\cite{robertlove2010}

\begin{lstlisting}
  struct thread_info {
        struct task_struct    *task;
        struct exec_domain    *exec_domain;
        unsigned long         flags;
        unsigned long         status;
        __u32                 cpu;
        __s32                 preempt_count;
        mm_segment_t          addr_limit;
        struct restart_block  restart_block;
        unsigned long         previous_esp;
        __u8                  supervisor_stack[0];
};
\end{lstlisting}

Compared to the \texttt{task\_struct}, this is a realitively simple data structure.
It has a pointer to the \texttt{task\_struct} that is unique to that thread,
along with a few extra flags and sets of information that seperate threads from
processes.

  \subsection{CPU Scheduling}
Linux ships with a few different schedulers by default, including a real time
scheduler.\cite{redhat2016} The default scheduler for Linux is the CFS, or
the Completely Fair Scheduler.
The CFS is only fair in so far as it tries its best to split the processors time
into equal bits, 1/n, where n is the number of processes currently running. The
CFS weights this with process priority, or nice value, and the higher the weight
the larger percentage of processor time that process gets. So it's not completely
fair, but that is the idea behind it.

One of the major problems with this system is that a very large chunk of processes
creates incredibly small slices of CPU time. For example, if there were six-thousand
processes and/or threads that all had the same weight, then the cpu slice for
each process would be 1/6000. There is a point where that slice is not large
enough to even complete the context switch required to start executing the process
again. The CFS handles this by implementing a floor on slice size, known as
\textit{minimum granularity}.\cite{robertlove2010} With this system, every process
will always have a minimum amount of run time, but the system can become very
slow once this size is reached.

% https://www.kernel.org/doc/Documentation/scheduler/sched-design-CFS.txt
% This is a good resource for indepth analysis of the scheduler

\section{Windows}
Windows is a very different beast. The process and thread structure are quite a
bit different from Linux. The CPU scheduler Windows uses has the same basic idea
as the CFS, but varies greatly when it comes to how it handles priority and division
of CPU time.
  \subsection{Processes}
Processes in Windows are similiar to Linux in only one way: the both have a
process ID. Otherwise, processes in Windows handle very differently than Linux.
Here is the basic process struct in Windows, \texttt{\_PROCESS\_INFORMATION}:
\cite{msprocstruct2016}

\begin{lstlisting}
  typedef struct _PROCESS_INFORMATION {
    HANDLE hProcess;
    HANDLE hThread;
    DWORD  dwProcessId;
    DWORD  dwThreadId;
  } PROCESS_INFORMATION, *LPPROCESS_INFORMATION;
\end{lstlisting}

The \texttt{HANDLE} data type is a 32-bit unique identifier stored in the Windows
kernel to identify individual processes.\cite{mshandle2016} The \texttt{hThread}
points to the primary thread a process is executing on. These first two of these
identifying fields are used internally by the kernel to specify what functions
are performing operations on the process and thread objects.

The second set of identifiers, \texttt{dwProcessId} and \texttt{dwThreadId} are
used as unique identifiers for the process and thread themselves. These are the
equivalent of the PID used by the the Linux \texttt{tast\_struct}.

  % https://msdn.microsoft.com/en-us/library/windows/desktop/ms681917(v=vs.85).aspx
  \subsection{Threads}
  \subsection{CPU Scheduling}
\section{FreeBSD}
  \subsection{Processes}
  \subsection{Threads}
  \subsection{CPU Scheduling}
  % https://msdn.microsoft.com/en-us/library/windows/desktop/ms685100(v=vs.85).aspx
\clearpage
\section{Appendix A}
The \texttt{task\_struct}:\cite{linustorvalds2016}
\begin{lstlisting}
  struct task_struct {
/* these are hardcoded - don't touch */
  volatile long        state;          /* -1 unrunnable, 0 runnable, >0 stopped */
  long                 counter;
  long                 priority;
  unsigned             long signal;
  unsigned             long blocked;   /* bitmap of masked signals */
  unsigned             long flags;     /* per process flags, defined below */
  int errno;
  long                 debugreg[8];    /* Hardware debugging registers */
  struct exec_domain   *exec_domain;
/* various fields */
  struct linux_binfmt  *binfmt;
  struct task_struct   *next_task, *prev_task;
  struct task_struct   *next_run,  *prev_run;
  unsigned long        saved_kernel_stack;
  unsigned long        kernel_stack_page;
  int                  exit_code, exit_signal;
  /* ??? */
  unsigned long        personality;
  int                  dumpable:1;
  int                  did_exec:1;
  int                  pid;
  int                  pgrp;
  int                  tty_old_pgrp;
  int                  session;
  /* boolean value for session group leader */
  int                  leader;
  int                  groups[NGROUPS];
  /*
   * pointers to (original) parent process, youngest child, younger sibling,
   * older sibling, respectively.  (p->father can be replaced with
   * p->p_pptr->pid)
   */
  struct task_struct   *p_opptr, *p_pptr, *p_cptr,
                       *p_ysptr, *p_osptr;
  struct wait_queue    *wait_chldexit;
  unsigned short       uid,euid,suid,fsuid;
  unsigned short       gid,egid,sgid,fsgid;
  unsigned long        timeout, policy, rt_priority;
  unsigned long        it_real_value, it_prof_value, it_virt_value;
  unsigned long        it_real_incr, it_prof_incr, it_virt_incr;
  struct timer_list    real_timer;
  long                 utime, stime, cutime, cstime, start_time;
/* mm fault and swap info: this can arguably be seen as either
   mm-specific or thread-specific */
  unsigned long        min_flt, maj_flt, nswap, cmin_flt, cmaj_flt, cnswap;
  int swappable:1;
  unsigned long        swap_address;
  unsigned long        old_maj_flt;    /* old value of maj_flt */
  unsigned long        dec_flt;        /* page fault count of the last time */
  unsigned long        swap_cnt;       /* number of pages to swap on next pass */
/* limits */
  struct rlimit        rlim[RLIM_NLIMITS];
  unsigned short       used_math;
  char                 comm[16];
/* file system info */
  int                  link_count;
  struct tty_struct    *tty;           /* NULL if no tty */
/* ipc stuff */
  struct sem_undo      *semundo;
  struct sem_queue     *semsleeping;
/* ldt for this task - used by Wine.  If NULL, default_ldt is used */
  struct desc_struct *ldt;
/* tss for this task */
  struct thread_struct tss;
/* filesystem information */
  struct fs_struct     *fs;
/* open file information */
  struct files_struct  *files;
/* memory management info */
  struct mm_struct     *mm;
/* signal handlers */
  struct signal_struct *sig;
#ifdef __SMP__
  int                  processor;
  int                  last_processor;
  int                  lock_depth;     /* Lock depth.
                                          We can context switch in and out
                                          of holding a syscall kernel lock... */
#endif
};
\end{lstlisting}
\clearpage
\section{Bibliography}

\bibliographystyle{IEEEtran}
\bibliography{writing_1}

\end{document}
