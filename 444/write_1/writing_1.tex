\documentclass[10pt,letterpaper,onecolumn,draftclsnofoot]{IEEEtran}
\usepackage[margin=0.75in]{geometry}
\usepackage{listings}
\usepackage{color}
\usepackage{longtable}
\usepackage{tabu}
\definecolor{dkgreen}{rgb}{0,0.6,0}
\definecolor{gray}{rgb}{0.5,0.5,0.5}
\definecolor{mauve}{rgb}{0.58,0,0.82}

\lstset{frame=tb,
  language=C,
  columns=flexible,
  numberstyle=\tiny\color{gray},
  keywordstyle=\color{blue},
  commentstyle=\color{dkgreen},
  stringstyle=\color{mauve},
  breaklines=true,
  breakatwhitespace=true,
  tabsize=4
}

\begin{document}
\begin{titlepage}
  \title{CS 444 - Spring 2016 - Writing Assignment 1}
  \author{Cody Malick\\
  \texttt{malickc@oregonstate.edu}}
  \date{April 21, 2016}
  \maketitle
  \vspace*{4cm}
  \begin{abstract}
      \noindent Understanding how an operating system handles processes, threads,
      and CPU scheduling are important topics for an aspiring kernel developer.
      In this report, we will cover how Linux, Windows, and FreeBSD handle these
      important structures in their respective kernels, and contrast them.
  \end{abstract}
\end{titlepage}

\tableofcontents
\clearpage
\section{Introduction}
This report contains descriptions on how each of the operating systems, Linux,
Windows, and FreeBSD handle processes, threads, and CPU scheduling. A lot can be
learned about the design decisions of the minds behind the operating system by
how they handle these core constructs.

I will start by describing how Linux handles these, as Linux is what we will be
thoroughly examining this term.

\section{Linux}
  \subsection{Processes}
  \subsection{Threads}
  \subsection{CPU Scheduling}

\section{Windows}
  \subsection{Processes}
  \subsection{Threads}
  \subsection{CPU Scheduling}
\section{FreeBSD}
  \subsection{Processes}
  \subsection{Threads}
  \subsection{CPU Scheduling}
  
\clearpage
\section{Bibliography}
\cite{johnm.2014}
\bibliographystyle{IEEEtran}
\bibliography{writing_1}

\end{document}
