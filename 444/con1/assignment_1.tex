\documentclass[10pt,letterpaper]{article}
\usepackage[margin=0.75in]{geometry}
\usepackage{listings}
\usepackage{color}

\definecolor{dkgreen}{rgb}{0,0.6,0}
\definecolor{gray}{rgb}{0.5,0.5,0.5}
\definecolor{mauve}{rgb}{0.58,0,0.82}

\lstset{frame=tb,
  language=C,
  columns=flexible,
  numberstyle=\tiny\color{gray},
  keywordstyle=\color{blue},
  commentstyle=\color{dkgreen},
  stringstyle=\color{mauve},
  breaklines=true,
  breakatwhitespace=true,
  tabsize=3
}

\begin{document}
\begin{titlepage}
  \title{CS 444 - Assignment 1}
  \author{Cody Malick\\
  \texttt{malickc@oregonstate.edu}}
  \date{\today}
  \maketitle

  \begin{abstract}
      This is my abstract section
  \end{abstract}

\end{titlepage}

\tableofcontents
\clearpage

\section{Concurrency Assignment 1}
  \subsection{Assignment Purpose}
    I think this assignment was used a primer for working with the kernel, where
    multithreading and shared resource management are important. Being able to
    write programs that don't deadlock or cause race conditions is important
    when working with a kernel. \\

  \subsection{Problem Approach}
    The over assignment wasn't that difficult to conceptualize. A producer and
    a consumer were needed, and the synchronization construct could be a simple
    data structure.\\

    \noindent I used a queue structure for my buffer. The main reason was
    simplicity, first-in first-out made sense in this situation. I thought about
    using a stack, but that created a problem where the first message may never
    be reached once burried under new messages.\\

    \noindent I used a very simple layout for the buffer: an array of the message
    struct, and an int to track the current size of the buffer.

    \begin{lstlisting}
      struct message {
      	int number;
      	int wait_period;
      };

      /*buffer is a basic queue, first in, first out implementation*/
      struct buffer {
      	struct message b[32];
      	int size;
      };
    \end{lstlisting}

    \noindent The queue has two mutator functions: push and pop. Push would put
    a new message on the queue, while pop would take the first message out, and
    move all following messages forward.\\

    \noindent To handle the producer function, I spawn my threads using
    \texttt{pthread\_create()} and handing the threads a producer function and consumer
    function. Once the threads enter these functions, they enter the infinite
    loops that are the assigned program. Here are the producer and consumer
    functions I used:

    \begin{lstlisting}
      void *producer_function(void *ptr)
      {
      	/*cast void pointer to buffer pointer*/
      	struct buffer *buff = (buffer *)ptr;
      	struct message item;

      	char prod[ ] = "Producer";

      	while(1) {
      		/*check if buffer is full*/
      		if(buff->size < (sizeof(buff->b)/sizeof(struct message))) {
      			pthread_mutex_lock(&mutex);
            /*UP, LO, CUP, CLO, PUP, and PLO, are constants declared to avoid magic numbers*/
      			item.number = rand_gen(UP, LO);
      			item.wait_period = rand_gen(CUP, CLO);

      			buff_push(buff, item);

      			print_column(prod, item.number, item.wait_period, buff->size);
      			pthread_mutex_unlock(&mutex);

      			sleep(rand_gen(PUP, PLO));
      			}
      		}
      }

      void *consumer_function(void *ptr)
      {
      	/*cast void pointer to buffer pointer*/
      	struct buffer *buff = (buffer*)ptr;
      	struct message item;

      	char cons[ ] = "Consumer";

      	int wait_time = 0;
      	while(1) {

      		if(buff->size > 0) {

      			pthread_mutex_lock(&mutex);

      			item = buff->b[0];
      			buff_pop(buff);

      			print_column(cons, item.number, item.wait_period, buff->size);
      			pthread_mutex_unlock(&mutex);

      			sleep(item.wait_period);
      		}
      	}
      }
    \end{lstlisting}

    \noindent The producer and consumer functions each lock the shared buffer
    with \texttt{pthread\_mutex\_lock(\&mutex)} and unlock with
    \texttt{pthread\_mutex\_unlock(\&mutex)} when they're done accessing it.
    I decided to include the \texttt{printf()} statement within the mutex locks
    so that I could actively print out the status of the queue's size.\\

    \noindent To handle the rdrand requirement, I took code from Intel's online
    guide to rdrand \cite{johnm.2014}. Using that code, I had to do a check in
    the main function and set a global flag, \texttt{int rdrand}, so that when
    my random function \texttt{rand\_gen()} is called it knows which generator
    to use. If rdrand is not supported, it calls Mersenne Twister with the
    \texttt{genrand\_int32()} function call. Here is my random generator function:

    \begin{lstlisting}
      int rand_gen(int upr_bound, int lwr_bound)
      {
      	if(rdrand) {
      		/*rdrand required unsigned 32 bit int*/
      		uint32_t num = 0;
      		if(rdrand32_step(&num))
      			return abs((int)num) % upr_bound + lwr_bound;
      	}
      	/*abs to solve negative numbers from generator*/
      	return abs((int)genrand_int32()) % upr_bound + lwr_bound;
      }
    \end{lstlisting}

    \noindent Lastly, to show that program is operating properly, my output does
    the following: lists if it's the producer or consumer, what the randomly
    generated number is, what the wait time of that message is, and how many items
    are in the buffer after that production or consumption is complete. \\

    \noindent Example: \\

    \noindent \texttt{Producer | Number:587 | Wait: 6 | buffer:  1 |} \\
    \texttt{Consumer | Number:587 | Wait: 6 | buffer:  0 |}

  \subsection{Testing}
  I did some manual testing to ensure that my pthread implementation was working.
  The first was to run the program multiple times and see that different printf
  statements printed at different times, a good indicator that the threads were
  in fact spawning and running properly. \\

  \noindent The other manual test I did was to move my mutex locks outside my
  producer and consumer function infinite loops. When I did that, in the producer's
  case, the producer would never release the mutex on the buffer, and the consumer
  would never be able to access the buffer. In the opposite case, the producer
  would produce zero or one messages, put them in the buffer, and the consumer
  would then lock the buffer for the rest of the program runtime. This resulted
  in nothing happening as the producer couldn't put anything in the buffer, and
  the consumer would never consume.

  \subsection{What did I learn?}
  I learned quite a bit in this assignment! I learned how to use pthreads (as this
  was not covered in my OS I class), how to use mutex to control which thread
  has access to a resource, how to implement some basic inline assembly, and
  gave myself a basic refresher on the C language and its caveats. \\

  \noindent I also learned the basics needed to build and implement a linux kernel,
  how to connect to and debug a Quemu VM, and how to use \LaTeX{}!

\section{Quemu Flags}
  The command we are told to use is the following: \\

  \noindent \texttt{qemu-system-i386 -gdb tcp::5500086 -S -nographic -kernel
      bzImage-qemux86.bin -drive file=core-\\image-lsb-sdk-qemux86.ext3,if=virtio
      -enable-kvm -net none -usb -localtime --no-reboot --append\\ "root=/dev/vda
      rw console=ttyS0 debug"}
\section{Work Log}

\section{Command Log}

\section{Version Control Log}

\clearpage

\bibliographystyle{IEEEtran}
\bibliography{assignment_1}

\end{document}
