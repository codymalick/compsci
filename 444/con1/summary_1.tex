\documentclass[10pt,letterpaper]{article}

\usepackage[margin=0.75in]{geometry}

\begin{document}

  \title{Weekly Summary 1}
  \author{Cody Malick\\
  \texttt{malickc@oregonstate.edu}}
  \date{\today}
  \maketitle

    In Robert Love's first two chapters, \textit{Introduction to the Linux
    Kernel} and \textit{Getting Started with the Kernel} (\date{2010}), Love
    explains that Linux kernel development is not as daunting as it seems.
    Love does this by first explaining what a kernel is, how it has evolved, and
    what knowledge and tools are needed to begin contributing to its
    development. Love's purpose in these chapters is to get the reader to take
    the first steps necessary to start working with the kernel, in order to
    reduce the barrier of entry into kernel development. The in-depth
    explanations and definitions of different parts of the kernel, as well
    as the short descriptions of certain features of kernel development, imply
    that Love is writing for aspiring kernel developers.

\end{document}
