\documentclass[10pt,letterpaper]{article}

\usepackage[margin=0.75in]{geometry}

\begin{document}

  \title{Weekly Summary 9}
  \author{Cody Malick\\
  \texttt{malickc@oregonstate.edu}}
  \date{May 29, 2016}
  \maketitle

    In Robert Love's tenth and thirteenth chapters, \textit{Kernel Synchronization Methods}
    and \textit{The Virtual File System} (\date{2010}), Love explains
    that the kernel uses synchronization structures and atomic operations to
    prevent race conditions, and the VFS to abstract the labor of manual file
    system tracking from the user. 
    Love does this by explaining how operations are made uninterruptable using
    atomic types, how synchronization is maintained using semaphores and spinlocks,
    and how the four major building blocks of the VFS (superbloc, inode, dentry, file)
    are used to make the user's life easier.
    Love's purpose for explaining these is to help the reader understand how
    the kernel manages to stay synchronized with its many threads, and how we use the
    VFS structures and their operations to manage the file system.
    The in-depth explanations of kernel synchronization structures and VFS structures
    and their operations, imply that Love is writing for aspiring kernel developers.

\end{document}
