\documentclass[10pt,letterpaper,onecolumn,draftclsnofoot]{IEEEtran}
\usepackage[margin=0.75in]{geometry}
\usepackage{listings}
\usepackage{color}
\usepackage{longtable}
\usepackage{tabu}
\definecolor{dkgreen}{rgb}{0,0.6,0}
\definecolor{gray}{rgb}{0.5,0.5,0.5}
\definecolor{mauve}{rgb}{0.58,0,0.82}

\lstset{frame=tb,
  language=C,
  columns=flexible,
  numberstyle=\tiny\color{gray},
  keywordstyle=\color{blue},
  commentstyle=\color{dkgreen},
  stringstyle=\color{mauve},
  breaklines=true,
  breakatwhitespace=true,
  tabsize=4
}

\begin{document}
\begin{titlepage}
  \title{CS 444 - Spring 2016 - Assignment 4}
  \author{Cody Malick\\
  \texttt{malickc@oregonstate.edu}}
  \date{June 2, 2016}
  \maketitle
  \vspace*{2cm}
  \begin{abstract}
      \noindent This report details implementation details for the fourth assignment.
      It contains a brief description of how I planned to implement the assignment,
      the work log, and my response to the questions asked on the assignment. \end{abstract}

\end{titlepage}

\tableofcontents
\clearpage

\section{Assignment 4}
  \subsection{Design}
	My approach for this problem is fairly straightforward: I'm going to change
	how the individual block allocator iterates over each block from first to
	best fit, then iterate over every available block and find first best fit
	over all of them. I plan on measuring efficiency by counting the number of
	blocks that each method yields, and graphing that.

	\section{Version Control and Work Log}
	Created with gitlog2latex
	\begin{center}
		\begin{longtabu} to \textwidth {|
		    X[4,l]|
		    X[3,c]|
		    X[8,l]|
		    X[4,1]|}
		    \hline
		    \textbf{Author} & \textbf{Date} & \textbf{Message} & \textbf{Time Spent} \\ \hline
		codymalick & 2016-06-02 & added variable, not working & 4 hours \\ \hline
		codymalick & 2016-05-15 & added some code, not working as of yet & 3 hours \\ \hline
		codymalick & 2016-05-15 & added design document & 1 hour \\ \hline
	\end{longtabu}
	\end{center}

\section{Questions}
	\subsection{Assignment Purpose}
	I believe the purpose of this assignment was to understand how memory is allocated,
	why it is an important problem to solve, and why best fit, although more efficient,
	is much slower than first fit. If we wanted to work with this interace in the kernel,
	understanding these basic concepts is important.

	\subsection{Problem Approach}
	I approached the problem initially the way I had planned in my design. I 
	wrote the code that found the best fit solution first, but I couldn't track down
	the reason I kept getting a kernel panic. I wrote some code to sort inside the
	\texttt{slob\_page\_alloc()} function, but I never saw if it worked properly
	because of my kernel panic. Because of the kernel panic, I also coudln't get the
	system call part of the assignment working.

	\subsection{Testing}
	Testing would be straightforward: If the VM ran, and was noticably slower, then
	the best fit implementation was working. We could also get exact metrics based
	off of the amount of memory that was used (lower for best fit, higher for first).

	\subsection{What did I learn?}
	Although I did not successfully finish the assignment, I learned how to get a custom
	system call working, how to work with the a simple list of blocks, and learned how
	different approaches to memory management are handled.
\end{document}
