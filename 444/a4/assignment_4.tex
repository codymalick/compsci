\documentclass[10pt,letterpaper,onecolumn,draftclsnofoot]{IEEEtran}
\usepackage[margin=0.75in]{geometry}
\usepackage{listings}
\usepackage{color}
\usepackage{longtable}
\usepackage{tabu}
\definecolor{dkgreen}{rgb}{0,0.6,0}
\definecolor{gray}{rgb}{0.5,0.5,0.5}
\definecolor{mauve}{rgb}{0.58,0,0.82}

\lstset{frame=tb,
  language=C,
  columns=flexible,
  numberstyle=\tiny\color{gray},
  keywordstyle=\color{blue},
  commentstyle=\color{dkgreen},
  stringstyle=\color{mauve},
  breaklines=true,
  breakatwhitespace=true,
  tabsize=4
}

\begin{document}
\begin{titlepage}
  \title{CS 444 - Spring 2016 - Assignment 3}
  \author{Cody Malick\\
  \texttt{malickc@oregonstate.edu}}
  \date{May 16, 2016}
  \maketitle
  \vspace*{2cm}
  \begin{abstract}
      \noindent This report details implementation details for the third assignment.
      It contains a brief description of how I planned to implement the assignment,
      the work log, and my response to the questions asked on the assignment. \end{abstract}

\end{titlepage}

\tableofcontents
\clearpage

\section{Assignment 3}
  \subsection{Design}
	My approach for this problem is fairly straightforward: I'm going to use
	the sbd (simple block device) driver as a skeleton for my code, and then
	I simply plan on modifying the read and write functions to encrypt and
	decrypt as necessary. I plan on using AES (Advanced Encryption Standard)
	to encrypt and decrypt the data. I will also use a seperate read and write
	key so as to demostrate the data is unreadable with the wrong key.

\section{Version Control and Work Log}
Created with gitlog2latex
\begin{center}
\begin{longtabu} to \textwidth {|
    X[4,l]|
    X[3,c]|
    X[8,l]|
    X[4,1]|}
    \hline
    \textbf{Author} & \textbf{Date} & \textbf{Message} & \textbf{Time Spent} \\ \hline
codymalick & 2016-05-16 & assignment complete & 2 hours \\ \hline
codymalick & 2016-05-15 & encryption working & 3 minutes \\ \hline
codymalick & 2016-05-15 & progress & 1.5 minutes \\ \hline
codymalick & 2016-05-12 & ramdisk working & 2 hours \\ \hline
\end{longtabu}
\end{center}

\section{Questions}
	\subsection{Assignment Purpose}
	I believe the purpose of this assignment is to give student experience working
	with modules, as well as touching the crypto API, and working with the block
	layer directly. Working with modules allows students to directly implement
	their own changes into Linux, as well as give them the tools they need to
	directly contribute to the feature set of Linux.

	\subsection{Problem Approach}
	I approached the problem initially the way I had planned in my design. I 
	acquired my basic block device driver using sbd. After getting the basic
	ramdisk working, I then started readin up on the crypto API, trying to find
	a basic implementation of encryption and decryption.

	I intially tried to directly call some of the Linux crypto files, which turned
	out to be a mistake. I got all sorts of wonderful errors when I was doing that.
	I then figured out how to call the \texttt{crypto.h} and had a much better experience.
	Once I had actually figured out how to call the crypto API using 
	\texttt{crypto\_alloc\_cipher}, and the other calls, actually
	using the crypto wasn't that bad.
	
	\subsection{Testing}
	Testing this poject was very straightforward. I manually print out the data
	from the buffer. The read data is gibberish before decryption. When I read from
	the file directly, it's readable. I also added two keys for my encrypt
	and decrypt functions, so my module's paramters are easily adjustable.

	\subsection{What did I learn?}
	I learned quite a bit on this assignment. It was a challenge!

	I learned how to build and inject a module, call some basic crypto API 
	functions, and build and use that module in the VM. I enjoyed this assignment.

\end{document}
