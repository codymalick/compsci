\documentclass[10pt,letterpaper]{article}

\usepackage[margin=0.75in]{geometry}

\begin{document}

  \title{Weekly Summary 4}
  \author{Cody Malick\\
  \texttt{malickc@oregonstate.edu}}
  \date{April 23, 2016}
  \maketitle

    In Robert Love's fifth and sixth chapters, \textit{Kernel Data Structures}
    and \textit{Interrupts and Interrupt Handlers} (\date{2010}), Love explains
    that the reader should use the kernel data structures, why you should use
    them, and how the kernel handles interrupts.
    Love does this by explaining the basic data structures the kernel uses and
    examples of when to use each, and by explaining Linux uses interrupts
    and interrupt handlers to communicate between hardware and the kernel.
    Love's purpose for explaining these is to help the reader avoid reimplementing
    tools that already exist, and understand how interrupts interface with the
    kernel to quickly execute requests from hardware.
    The in-depth explanations of the kernel data structures, interrupt structure,
    and handler implementation, imply that Love is writing for aspiring kernel
    developers.

\end{document}
