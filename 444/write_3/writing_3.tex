\documentclass[10pt,letterpaper,onecolumn,draftclsnofoot]{IEEEtran}
\usepackage[margin=0.75in]{geometry}
\usepackage{listings}
\usepackage{color}
\usepackage{longtable}
\usepackage{tabu}
\definecolor{dkgreen}{rgb}{0,0.6,0}
\definecolor{gray}{rgb}{0.5,0.5,0.5}
\definecolor{mauve}{rgb}{0.58,0,0.82}

\lstset{frame=tb,
  language=C,
  columns=flexible,
  numberstyle=\tiny\color{gray},
  keywordstyle=\color{blue},
  commentstyle=\color{dkgreen},
  stringstyle=\color{mauve},
  breaklines=true,
  breakatwhitespace=true,
  tabsize=4
}

\begin{document}
\begin{titlepage}
  \title{CS 444 - Spring 2016 - Writing Assignment 3}
  \author{Cody Malick\\
  \texttt{malickc@oregonstate.edu}}
  \date{May 10, 2016}
  \maketitle
  \vspace*{4cm}
  \begin{abstract}
      \noindent Understanding how an operating system handles interrupts is key
        to understanding the overall flow of data through an OS. In this report,
         we will cover how Linux, Windows, and FreeBSD handle this important job
         in their respective kernels, and contrast them.
  \end{abstract}
\end{titlepage}

\tableofcontents
\clearpage
\section{Introduction}
    Interrupts are how hardware communicates with the CPU, letting it know it
    has information of some kind to share with it. Interrupts are taken care
    of by interrupt handlers. This report will examine what the structure of
    interrupts are in Linux, Windows, and FreeBSD, the general procedure of
    how they are handled, and where these interrupt handlers live in their
    respective OS. 
 % Actually write an introduction, including why the reader should care
\section{Linux}
  \subsection{Interrupts}
  Interrupts are how a piece of hardware communicates with the cpu, letting it
  know that it has some data ready for it. An interrupt must have an interrupt
  handler associated with it in order to have the communication handled properly.
  These interrupt handlers are contained in a devices driver. This is pretty
  universal across operating systems. What we are going to look at specifically
  is the structure of interrupts that Linux handles. \cite{robertlove2010}

  \subsection{Interrupt Structure}
  Interrupts are split into two parts in Linux: top halves, and bottom halves.
  These names are quite misleading as top halves and bottom halves are not halves
  at all. The top half is usually closer to a fifth or even an eighth. The reason
  for this is that interrupt handlers have to be very fast! Interrupts, as they
  are aptly named, stop everything the cpu is doing in order to communicate with
  hardware. When this interrupt is called, all the code in the top half of the
  interrupt has to be handled extremely quickly. Because of this, only time sensitive
  information can be stored in the top half, and all information in the top half
  is handled by the interrupt handler.

  The bottom half, on the other hand, can contain any extra information needed
  to process the request from the hardware. Because this information is not deemed
  time critical, it is queued up with the other requests that the CPU needs to handle
  and is put off until its time to be processed has come. \cite{robertlove2010}

\section{Windows}
 \subsection{Interrupts}
 \subsection{Drivers}
 Drivers are important in every OS, but in Windows they have a very specific
 structure. Drivers have a few responsibilities besides handling interrupts as
 they do in Linux. They have the responsibility to start IO requests to the hardware,
 handle plug-and-play behavior, cancel IO routines, unload routines, and more.
 Drivers are much more of entire modules instead of specifically interrupt handlers
 as they are in Linux. \cite{internals2}

 This view of drivers makes the handling of IO much more of the drivers responsibility
 than it does in Linux. In Linux, the driver simply handles the interrupt and hands
 other information off to the OS. In Windows, the driver is more robust and has
 more responsibility on how things are communicated.

\section{FreeBSD}
 \subsection{Interrupts}
 Interrupts are handled almost identically to Linux's interrupt setup. Each
 interrupt must be registered via a device driver. FreeBSD has two types of
 interrupts: hardware interrupts and software interrupts. These are the equivalent
 of Linux's top and bottom halves. If a piece of information is time critical,
 it is put in a hardware interrupt, otherwise they are handed to a software
 interrupt. \cite{freebsd2016}
 \subsection{Interrupt Handler}

\section{Conclusion}

\clearpage
\section{Appendix A - Linux Structs}
All following structs have been pulled from the classroom text. \cite{robertlove2010}
\begin{lstlisting}

for future code samples

\end{lstlisting}

\section{Appendix B - Windows Structs}
\begin{lstlisting}

for future code samples
\end{lstlisting}
\section{Appendix C - FreeBSD Structs}
\begin{lstlisting}
for future code samples

\end{lstlisting}

\section{Bibliography}
\bibliographystyle{IEEEtran}
\bibliography{writing_3}

\end{document}
