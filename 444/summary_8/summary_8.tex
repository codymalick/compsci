\documentclass[10pt,letterpaper]{article}

\usepackage[margin=0.75in]{geometry}

\begin{document}

  \title{Weekly Summary 8}
  \author{Cody Malick\\
  \texttt{malickc@oregonstate.edu}}
  \date{May 22, 2016}
  \maketitle

    In Robert Love's eleventh and sixteenth chapters, \textit{Timers and Time Management}
    and \textit{The Page Cache and Page Writeback} (\date{2010}), Love explains
    that the kernel uses timers to maintain an internal state of relative time, and
    page caching to increase performance and reduce overall disk IO. 
    Love does this by explaining how time is maintained in the kernel using
    ticks and the timer interrupt handler to maintain internal state, and by
    explaining how Linux caches IO request results in RAM to increase performance.
    Love's purpose for explaining these is to help the reader understand how
    the kernel manages time and how the user can use timers in their own code,
    and to show the reader why caching is important for performance and reduced
    power usage. The in-depth explanations of timer functionality and how they
    are used, as well as the detailed explaination of caching, imply that Love
    is writing for aspiring kernel developers.

\end{document}
