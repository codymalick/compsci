\documentclass[10pt,letterpaper,onecolumn,draftclsnofoot]{IEEEtran}
\usepackage[margin=0.75in]{geometry}
\usepackage{listings}
\usepackage{color}
\usepackage{longtable}
\usepackage{tabu}
\definecolor{dkgreen}{rgb}{0,0.6,0}
\definecolor{gray}{rgb}{0.5,0.5,0.5}
\definecolor{mauve}{rgb}{0.58,0,0.82}

\lstset{frame=tb,
  language=C,
  columns=flexible,
  numberstyle=\tiny\color{gray},
  keywordstyle=\color{blue},
  commentstyle=\color{dkgreen},
  stringstyle=\color{mauve},
  breaklines=true,
  breakatwhitespace=true,
  tabsize=4
}

\begin{document}
\begin{titlepage}
  \title{CS 444 - Spring 2016 - Assignment 2}
  \author{Cody Malick\\
  \texttt{malickc@oregonstate.edu}}
  \date{April 26, 2016}
  \maketitle
  \vspace*{2cm}
  \begin{abstract}
      \noindent The report details implementation details for the second assignment.
      It contains a brief description of how I planned to implement the assignment,
      the work log, and my response to the questions asked on the assignment. \end{abstract}

\end{titlepage}

\tableofcontents
\clearpage

\section{Assignment 2}
  \subsection{Design}
I plan to do the following to implement this I/O scheduler:
\begin{itemize}
	\item Implmement an insertion sort on the request queue, sort is
		accomplished by finding the shortest distance to any request in
		the queue
	\item Once the queue is sorted, sort what gets sent to the dispatch
		queue by selecting next closest seek distance
	\item Merge adjacent requests

\end{itemize}
\section{Version Control and Work Log}
Created with gitlog2latex
\begin{center}
\begin{longtabu} to \textwidth {|
    X[4,l]|
    X[3,c]|
    X[8,l]|
    X[4,1]|}
    \hline
    \textbf{Author} & \textbf{Date} & \textbf{Message} & \textbf{Time Spent} \\ \hline
codymalick & 2016-04-27 & added merging & 2 hours \\ \hline
codymalick & 2016-04-27 & fixed if-else, updated .gitignore & 30 minutes \\ \hline
codymalick & 2016-04-27 & added files to rebase & 20 minutes \\ \hline
codymalick & 2016-04-26 & complete assignment 2 & 4 hours \\ \hline
codymalick & 2016-04-25 & shortest seek distance first working, next, get direction fixed  & 3 hours \\ \hline
codymalick & 2016-04-25 & renamed everything in sstf-iosched.c  & 20 minutes hours \\ \hline
codymalick & 2016-04-24 & added working qemu command & 3 hours \\ \hline
codymalick & 2016-04-24 & sstf showing up in vm, vm is operational & 1 hour \\ \hline
Cody Malick & 2016-04-24 & added files, working boot vm & 30 minutes \\ \hline
codymalick & 2016-04-23 & created sstf file & 1 hour \\ \hline
\end{longtabu}
\end{center}

\section{Questions}
	\subsection{Assignment Purpose}
	I believe the purpose of this assignment was to get the students familiarized
	with the function and operation of the IO elevator, how it interacts with the
	scheduler, and how different approaches behave.

	\subsection{Problem Approach}
	This was a very interesting assignment for me. Not only am I not familiar
	with the code in the kernel, as this is the first time we've actually worked
	with it, but it's my first time touching code at this level in the kernel.

	The first thing I did was research how the noop scheduler worked, not just
	overall, but internally. I then based my code on a working version of the
	noop scheduler. Before I did anything, I worked out how to get the renamed
	noop scheduler working with qemu, and once I had successfuly set the default
	scheduler for the kernel, I began work on the code.

	Once I had it working, I studied what exactly was needed for the assignment.
	I really struggled with the second merge. After drawing on a board for several
  hours I finally understood what was needed, and implemented it.

	After getting the sorting working, I worked on understanding how merge worked.
	Merging is a great idea, and the trouble was getting the actual merge working.
	The other issue was that merges occur once in a while. So testing is not
	consistent. Once I had figured out how to merge, I had to run a lot of IO
	to get my debugging statement to pop up in the VM.

	I ended up writing this assignment twice, once in the
	\texttt{dispatch\_request()} function, and once in the \
	texttt{add\_request()} function. The reason for this was the conversation
	that Kevin had with the class that the dispatch function shouldn't be touched.

	\subsection{Testing}
	Testing this poject was very straightforward. I added \texttt{printk()} inside
	every function that I modified and called. When I ran the vm with my scheduler
	set as the default, it was very obvious that my scheduler was being called.

	The other thing I had to confirm was that my requests were being queued properly.
	I did this by printing out whenever I iterated over the entire request queue.

	\subsection{What did I learn?}
	I learned quite a bit on this assignment. It was a challenge!

	I learned how IO schedulers function, how different implementations differ,
	and how to actually modify, implement, and patch a kernel to use a custom
	scheduler.

\end{document}
