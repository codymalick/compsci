\documentclass[10pt,letterpaper]{article}

\usepackage[margin=0.75in]{geometry}

\begin{document}

  \title{Weekly Summary 5}
  \author{Cody Malick\\
  \texttt{malickc@oregonstate.edu}}
  \date{April 30, 2016}
  \maketitle

    In Robert Love's eighth and twelfth chapters, \textit{Bottom Halves and Deffering Work}
    and \textit{Memory Management} (\date{2010}), Love explains
    that the kernel uses different functions and structures to manage memory,
    how they differ, and how deffering work in the second 'half' of an interrupt works.
    Love does this by explaining the basic structure of memory management in the
    kernel through pages and zones, what functions to use when allocating and freeing
    memory, and how softirqs and tasklets are used to pass additional data to the
    CPU. Love's purpose for explaining these is to help the reader understand how
    basic IO is handled between hardware and the cpu, and to educate the user on
    the basics of memory handling in the kernel using \texttt{kmalloc()} and
    \texttt{kfree()}. The in-depth explanations of the interrupt structure,
    memory pages and zones, and functions used, imply that Love is writing for
    aspiring kernel developers.

\end{document}
