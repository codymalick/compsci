\documentclass[10pt,letterpaper]{article}

\usepackage[margin=0.75in]{geometry}

\begin{document}

  \title{Weekly Summary 6}
  \author{Cody Malick\\
  \texttt{malickc@oregonstate.edu}}
  \date{May 7, 2016}
  \maketitle

    In Robert Love's fifteenth and seventeenth chapters, \textit{The Process Address Space}
    and \textit{Devices and Modules} (\date{2010}), Love explains
    that the kernel uses memory descriptors and virtual memory areas to manage
    process memory, and that the kernel can be augmented using modules.
    Love does this by explaining the how process address space is managed using the
    \texttt{mm\_struct} and associated functions, and by explaining how modules
    are built from hello world through passing parameters.
    Love's purpose for explaining these is to help the reader understand how
    the kernel manages process and thread memory, and to enable the reader
    to start working on his own extensions to the kernel using modules.
    The in-depth explanations of process address space structure and usage, as
    well as the basic outline of module creation, imply that Love is writing for
    aspiring kernel developers.

\end{document}
