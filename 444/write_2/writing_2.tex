\documentclass[10pt,letterpaper,onecolumn,draftclsnofoot]{IEEEtran}
\usepackage[margin=0.75in]{geometry}
\usepackage{listings}
\usepackage{color}
\usepackage{longtable}
\usepackage{tabu}
\definecolor{dkgreen}{rgb}{0,0.6,0}
\definecolor{gray}{rgb}{0.5,0.5,0.5}
\definecolor{mauve}{rgb}{0.58,0,0.82}

\lstset{frame=tb,
  language=C,
  columns=flexible,
  numberstyle=\tiny\color{gray},
  keywordstyle=\color{blue},
  commentstyle=\color{dkgreen},
  stringstyle=\color{mauve},
  breaklines=true,
  breakatwhitespace=true,
  tabsize=4
}

\begin{document}
\begin{titlepage}
  \title{CS 444 - Spring 2016 - Writing Assignment 1}
  \author{Cody Malick\\
  \texttt{malickc@oregonstate.edu}}
  \date{April 21, 2016}
  \maketitle
  \vspace*{4cm}
  \begin{abstract}
      \noindent Understanding how an operating system handles input and output
      to devices and how they are scheduled are important topics for an aspiring
      kernel developer. In this report, we will cover how Linux, Windows, and
      FreeBSD handle these important tasks in their respective kernels,
      and contrast them.
  \end{abstract}
\end{titlepage}

\tableofcontents
\clearpage
\section{Introduction}
 % Actually write an introduction, including why the reader should care
Every computer, whether it's a mobile phone or a digital watch, needs to be able
to read and write to storage. For general desktop computers, they have to be able
to communicate with a vast variety of devices from hard drives to MicroSD cards.
In this paper, we will discuss how Linux, Windows, and FreeBSD handle input and
output, how they make it efficient, and contrast them to each other.
\section{Linux}
  \subsection{Virtual Filesystem}
  At the top layer of our input-output system, we have the virtual filesystem,
  otherwise known as the VFS. The virtual filesystem allows user-space programs
  to use standard unix systems calls such \texttt{read()} and \texttt{write()}
  regardless of storage medium. This is setup is only possible because Linux
  adds a layer of abstraction around the low-level filesystem. Although we take
  such functionality for granted, it is a very important system to have.
  \cite{robertlove2010}

  The VFS provides an abstraction of the filesystem, or multiple filesystems
  through a few specific interfaces. Specifically, the VFS provides the
  following abstractions:
  \begin{description}
    \item Files: an ordered string of bytes.
    \item Dentries: Files that contain directory information. For example,
    '\textbackslash this\textbackslash is\textbackslash a\textbackslash path'
    has four dentries on the path to that directory. This is how the VFS contains
    directories as files.
    \item Inodes: Otherwise known as an 'index node,' these files contain metadata
    about other files such as permissions, size, owner, etc.
    \item Superblock: A file containing all the relevant information about a filesystem
    as a whole. This is a metadata file for an entire FS.
  \end{description}

  Each of these interfaces have sizable data structures, along with a corresponding
  operations struct that complements them, containing function pointers to give the
  VFS the great functionality it has.\cite{robertlove2010}

  These interfaces are critical to how we use Linux every day. Without them, we
  would have to make manual calls to the different filesystems that managed different
  devices, in the protocols they require. Sounds like a mess!

  \subsection{Block IO Layer}
  The Block IO layer is where the dirty work actually gets done. This layer
  manages how data is actual written to and retreived from the storage device.
  Any device that contains storage is considerd a block device to the kernel.
  Usually, these block devices contain segments called \textit{sectors}, which
  are the smallest size that data can be stored in. The standard size is usually
  512 bytes, but it's possible for these sizes to vary.

  After a sector, we have the \textit{block}. The block is the smallest addressable
  unit by the kernel. This is an abstraction imposed by the filesystem. The block
  can be the same size as a single sector, but not smaller, and a block must also
  contain a multiple of the sector size. If the sector size is 512 bytes, then the
  block cannot be 800 bytes, it must be 1024, or 2048 bytes. It cannot contain
  anything but whole sectors.

  The actual reading and writing is handled by buffers and buffer heads. These
  buffers contain blocks that they are mapped to while they are pending to read
  or write from any given block. Buffer heads are the kernels way of having a
  descriptor pointing at these buffers. A buffer head contains all the information
  needed for the kernel to control and manipulate these buffers. \cite{robertlove2010}

  \begin{lstlisting}
    struct buffer_head {
          unsigned long b_state; /* buffer state flags */
          struct buffer_head *b_this_page; /* list of page’s buffers */
          struct page *b_page; /* associated page */
          sector_t b_blocknr; /* starting block number */
          size_t b_size; /* size of mapping */
          char *b_data; /* pointer to data within the page */
          struct block_device *b_bdev; /* associated block device */
          bh_end_io_t *b_end_io; /* I/O completion */
          void *b_private; /* reserved for b_end_io */
          struct list_head b_assoc_buffers; /* associated mappings */
          struct address_space *b_assoc_map; /* associated address space */
          atomic_t b_count; /* use count */
    };
\end{lstlisting}
  As you can see from the above code, the \texttt{buffer\_head} struct contains
  all the information the kernel needs. It has \texttt{b\_state} to track the
  state of the buffer, \texttt{buffer\_head} to track the page's buffers,  a pointer
  the the page associated with this buffer, and more. With this struct, the kernel
  can have direct control over what information is handled in the Block IO layer.

  \subsection{Scheduling}
  Along with actually getting IO done, the kernel must decide what order these
  operations need to be done in. The kernel also needs to reduce the total
  number of requests by efficiently merging requests that can be merged. This is
  handled by the IO scheduler. There are many different approaches to different
  schedulers. It is a very fascinating area for that reason.

  For example the noop (no operations) scheduler is the most basic one out there.
  All it does is maintain a request queue that is a first-in first-out data structure. It's
  only other responsibility is merging requests that are adjacent to other requests.
  This scheduler is the default scheduler for solid state drives and any sort of
  flash memory. The reason for this being that solid state devices have no moving
  parts, no head constantly seeking for different sectors. All read and write locations
  are self-sorting, and instant to read and write from. This scheduler performs
  absolutely poorly, however, when confronted with a spinning drive.

  A spinning drive requires a scheduler to group requests by location. This is
  needed because if we simply serviced requests in a FIFO manner, then the head
  would spend enormous amounts of time seeking for sectors, and slow down the OS
  tremendously. For this reason, the default scheduler for disk devices on Linux
  is the CFQ scheduler.

  The Completely Fair Queuing IO Scheduler has the following functionality: it
  maintains a seperate qeuue for each process, it services these requests in a round
  robin fashion, and performs merging of adjacent requests. This works very well
  for a number of reasons. The first of which, is each process gets the IO it needs
  done in a fairly prompt manner, but also prevents any one process from being
  starved out of IO by another read or write heavy operation. Another is that
  if the system is a multiple user sytem, which almost every computer is these
  days, then each user will get their IO serviced in a prompt manner.

  This is a basic overview of how Linux handles IO. Next, we'll examine how Windows
  and FreeBSD handle these same tasks, and compare them to how Linux operates.

\section{Windows}
 \subsection{Virtual File System}
 \subsection{Block IO Layer}
 \subsection{Scheduling}
\section{FreeBSD}
 \subsection{Virtual File System}
 \subsection{Block IO Layer}
 \subsection{Scheduler}
clearpage
\section{Appendix A - Linux Structs}
All following structs have been pulled from the classroom text. \cite{robertlove2010}
\begin{lstlisting}


\end{lstlisting}

\section{Appendix B - Windows Structs}
\begin{lstlisting}


\end{lstlisting}
\section{Appendix C - FreeBSD Structs}
\begin{lstlisting}


\end{lstlisting}

section{Bibliography}
\bibliographystyle{IEEEtran}
\bibliography{writing_2}

\end{document}
