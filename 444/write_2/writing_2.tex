\documentclass[10pt,letterpaper,onecolumn,draftclsnofoot]{IEEEtran}
\usepackage[margin=0.75in]{geometry}
\usepackage{listings}
\usepackage{color}
\usepackage{longtable}
\usepackage{tabu}
\definecolor{dkgreen}{rgb}{0,0.6,0}
\definecolor{gray}{rgb}{0.5,0.5,0.5}
\definecolor{mauve}{rgb}{0.58,0,0.82}

\lstset{frame=tb,
  language=C,
  columns=flexible,
  numberstyle=\tiny\color{gray},
  keywordstyle=\color{blue},
  commentstyle=\color{dkgreen},
  stringstyle=\color{mauve},
  breaklines=true,
  breakatwhitespace=true,
  tabsize=4
}

\begin{document}
\begin{titlepage}
  \title{CS 444 - Spring 2016 - Writing Assignment 1}
  \author{Cody Malick\\
  \texttt{malickc@oregonstate.edu}}
  \date{April 21, 2016}
  \maketitle
  \vspace*{4cm}
  \begin{abstract}
      \noindent Understanding how an operating system handles input and output
      to devices and how I/O is scheduled are important topics for an aspiring
      kernel developer. In this report, we will cover how Linux, Windows, and
      FreeBSD handle these important tasks in their respective kernels, 
      and contrast them.
  \end{abstract}
\end{titlepage}

\tableofcontents
\clearpage
\section{Introduction}
 % Actually write an introduction, including why the reader should carev

\section{Linux}
  \subsection{Processes}
  \subsection{Threads}
\section{Windows}
 \subsection{Processes}
 \subsection{Threads}
 \subsection{CPU Scheduling}
section{FreeBSD}
 \subsection{Processes}
 \subsection{Threads}
 \subsection{CPU Scheduling}
clearpage
\section{Appendix A}
section{Bibliography}

\bibliographystyle{IEEEtran}
\bibliography{writing_2}

\end{document}
