\documentclass[10pt,letterpaper,onecolumn,draftclsnofoot]{IEEEtran}
\usepackage[margin=0.75in]{geometry}
\usepackage{listings}
\usepackage{color}
\usepackage{longtable}
\usepackage{tabu}
\definecolor{dkgreen}{rgb}{0,0.6,0}
\definecolor{gray}{rgb}{0.5,0.5,0.5}
\definecolor{mauve}{rgb}{0.58,0,0.82}

\lstset{frame=tb,
  language=C,
  columns=flexible,
  numberstyle=\tiny\color{gray},
  keywordstyle=\color{blue},
  commentstyle=\color{dkgreen},
  stringstyle=\color{mauve},
  breaklines=true,
  breakatwhitespace=true,
  tabsize=4
}

\begin{document}
\begin{titlepage}
  \title{CS 444 - Spring 2016 - Writing Assignment 5}
  \author{Cody Malick\\
  \texttt{malickc@oregonstate.edu}}
  \date{May 31, 2016}
  \maketitle
  \vspace*{4cm}
  \begin{abstract}
      \noindent Understanding how an operating system handles its file systems and
      virtual file systems, and how that helps make the user's life easier by abstracting
      the physical system is an important topic for an aspiring kernel developer.
      In this report, we will cover how Linux, Windows, and FreeBSD handle the
      VFS in their respective kernels.
  \end{abstract}
\end{titlepage}

\tableofcontents
\clearpage
\section{Introduction}
 % Actually write an introduction, including why the reader should care
As a general user, you don't want to have to deal directly with a computers
input and output systems. It would be tiresome and tedious. To solve this
problem, the file system and virtual file systems are wrappers around these
arduous tasks that make it simple, and user friendly. The VFS is the unification
of two ideas: making the user's life easy, while providing a generic interface
that allows different types of file systems the same interfaces. Following is
an examination of the FS and VFS implementations in Linux, Windows, and FreeBSD. 
\section{Linux}
  \subsection{File Systems}
  A file system, simply, is an abstraction of the IO layer that allows ease of
  use for the user in tracking, reading, and writing to disk. Instead of forcing
  the user to remember where a file is, what its size is, what type of file it
  is, etc., the file system provides interfaces that make this a, relatively,
  easy task. There are many different types of file systems. They are 
  differentiated by minimum and maximum file size, journaling capability, and
  maximum partition size. 
  \subsubsection{Linux Default File Systems}
  There are a few default file systems available in Linux. A few of them are
  all from the \texttt{EXT} family, like \texttt{EXT2}, \texttt{EXT3}, and
  \texttt{EXT4}. \texttt{EXT3} is considered the standard FS for stable Linux
  builds, but \texttt{EXT4} is the latest and greatest release from that family.
  The biggest difference between \texttt{EXT4} and its previous version is the
  maximum file and partition size.\cite{anthonyjsimon2015}
  One the great things about Linux is the number of options for file systems.
  If there are features you're looking for, but don't have, then you can pick
  one that works better for your needs. If you want a bleeding edge system, you
  can also grab your favorite choice from the internet. 

  \subsection{Virtual Filesystem}
  The VFS lies at the top layer of the OS input-output system. The virtual
  filesystem allows user-space programs to use standard unix systems calls
  such \texttt{read()} and \texttt{write()} regardless of storage medium or
  underlying file system. This is setup is only possible because Linux adds
  a layer of abstraction around the low-level filesystem. Although we take
  such functionality for granted, it is a very important system to have.
  The VFS is divide into two overall pieces: structures, and their associated
  operations. \cite{robertlove2010}

  \subsection{VFS Structures}
  The VFS provides an abstraction of the filesystem, or multiple filesystems
  through a few specific interfaces. Specifically, the VFS provides the
  following abstractions:
  \begin{description}
    \item Files: an ordered string of bytes.
    \item Dentries: Files that contain directory information. For example,
    '\textbackslash this\textbackslash is\textbackslash a\textbackslash path'
    has four dentries on the path to that directory. This is how the VFS contains
    directories as files.
    \item Inodes: Otherwise known as an 'index node,' these files contain metadata
    about other files such as permissions, size, owner, etc.
    \item Superblock: A file containing all the relevant information about a filesystem
    as a whole. This is a metadata file for an entire FS.
  \end{description}

  Each of these interfaces have sizable data structures, along with a corresponding
  operations struct that complements them, containing function pointers to give the
  VFS the great functionality it has.\cite{robertlove2010}

  These interfaces are critical to how we use Linux every day. Without them, we
  would have to make manual calls to the different filesystems that managed different
  devices, in the protocols they require. Sounds like a mess!

\section{Windows}
Windows does things quite a bit different than Linux as far as file systems go.
This comes as no surprise as Windows has a trend of doing its own thing. Because
it is not originally a Unix based or Unix inspired operating system, we get to
see a very different approach to solving similiar problems. In the following
section, we will examine the Windows file system, \texttt{NTFS}, and its virtual
file system layer.
  \subsection{NTFS}
  Windows has a few different file systems available by default. The native file
  system, however, is called \texttt{NTFS}, or New Technology File System. 
  \texttt{NTFS} is fairly comperable to \texttt{EXT4} in maximum partition and 
  file size. \texttt{NTFS} theoretically supports up to exabyte volume sizes, but
  Windows currently limits support to 256 terrabyte volume sizes with 64 kilobyte
  cluster sizes. \texttt{NTFS} has the normal array of features that standard file
  systems ship with, along with file and directory security, alternate data streams,
  file compression, symbolic and hard links, encryption, and transactional semantics.
  \cite{internals2}
  \subsection{NTFS Driver}
  Windows implements the virtual file system layer using the file system module itself.
  In this, Windows doesn't have a distinct VFS layer. It implements the file system
  and the VFS in one module, the \texttt{NTFS} module. This is a hard contract between
  Linux and Windows. Linux seperates the VFS layer entirely, and then has the FS plug
  into the VFS and integrate. If you wanted to plug another file system in to Windows,
  it would use a distinctly different set of interfaces from another. So if you wanted
  to use a \texttt{FAT32} file system type, it would not easily integrate with a 
  \texttt{NTFS} file system. \cite{internals2}

  Now that we've examined how Windows handles its filesystem layer, and saw how
  different its monolithic approach is Linux' abstracted layers approach, we will
  examine how FreeBSD handles these same concepts.

\section{FreeBSD}
  FreeBSD is very similar to Linux in how it handles many things. The file system
  is no exception to this rule. FreeBSD is very similiar to Linux in this department,
  and we examine the subtle differences it has.
 \subsection{The Fast File System}
 The Fast File System is FreeBSD's default file system. It is fairly standard as
 File systems come, and is well tested and stable. The FS has features like directory
 structure, file names, journaling, etc. The standard set of features. It does not
 do file and directory security, like Windows. Permissions are managed, but not access
 to specific files and directories.

 \subsection{Virtual File System}
 The VFS setup in FreeBSD has a few structs that are similiar to the Linux structs.
 Something interesting about the FreeBSD versus the Linux is that FreeBSD
 makes explicit statements about how a file's existence is defined. For example,
 a file in FreeBSD exists until no references or descriptors are open in the OS.
 Linux may have similiar functions, but the rules for these are not explicitly
 defined. Here are the basic objects that make up the FreeBSD basic IO system:
 \cite{freebsd2016}
 \begin{description}
   \item Files: A Linear array of bytes with at least one name. A file exists
   until all its names are explicitly deleted explicitly and no process holds a
   descriptor of that file. All IO devices are treated as files.
   \item Directory Entry: A file that contains information about itself and
   other files and directory entries.
   \item Pipes: Pipes are also a linear array of bytes, but are used exclusively
   for one direction data transfer through the use of an IO stream. If you need
   to read and write from a file at the same time, two pipes will need to be open.
   \item Socket: An object that is used for interprocess communication, it exists
   only as long as some process holds the descriptor for it.
 \end{description}

 This setup is almost identical to how Linux handles things, but with one big
 difference: other files sytems mounted inside of the virtual file system are treated simply
 as directories, not as whole units like Linux's superblocks. This simplifies
 the handling of the entire filesystem structure. A root directory of a filesystem
 is set so that the OS knows where everything starts.

 \section{Conclusion}
Linux, Windows, and FreeBSD all have their own way of doing things. Although Linux
and FreeBSD share a lot of the same ideas, they are unique. Windows is another beast
entirely with its modules and driver structures. Understanding how all these systems
work are important as a developer working in the each of their respective kernels.

\clearpage
\section{Appendix A - Linux Structs}
All following structs have been pulled from the classroom text. \cite{robertlove2010}
\begin{lstlisting}

for future code samples

\end{lstlisting}

\section{Appendix B - Windows Structs}
\begin{lstlisting}

for future code samples
\end{lstlisting}
\section{Appendix C - FreeBSD Structs}
\begin{lstlisting}
for future code samples

\end{lstlisting}

section{Bibliography}
\bibliographystyle{IEEEtran}
\bibliography{writing_5}

\end{document}
