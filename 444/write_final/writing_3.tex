% \documentclass[10pt,letterpaper,onecolumn,draftclsnofoot]{IEEEtran}
% \usepackage[margin=0.75in]{geometry}
% \usepackage{listings}
% \usepackage{color}
% \usepackage{longtable}
% \usepackage{tabu}
\definecolor{dkgreen}{rgb}{0,0.6,0}
\definecolor{gray}{rgb}{0.5,0.5,0.5}
\definecolor{mauve}{rgb}{0.58,0,0.82}

\lstset{frame=tb,
language=C,
columns=flexible,
numberstyle=\tiny\color{gray},
keywordstyle=\color{blue},
commentstyle=\color{dkgreen},
stringstyle=\color{mauve},
breaklines=true,
breakatwhitespace=true,
tabsize=4
}

% \begin{document}
  \begin{titlepage}
    \title{CS 444 - Spring 2016 - Writing Assignment 3}
    \author{Cody Malick\\
    \texttt{malickc@oregonstate.edu}}
    \date{May 10, 2016}
    \maketitle
    \vspace*{4cm}
    \begin{abstract}
      \noindent Understanding how an operating system handles interrupts is key
      to understanding the overall flow of data through an OS. In this report,
      we will cover how Linux, Windows, and FreeBSD handle this important job
      in their respective kernels, and contrast them.
    \end{abstract}
  \end{titlepage}

  \tableofcontents
  \clearpage
  \section{Introduction}
  Interrupts are how hardware communicates with the CPU, letting it know it
  has information of some kind to share with it. Once the interrupt is received,
  the CPU interrupts the kernel, handing off the information it has. Interrupts
  are taken care of by interrupt handlers. This report will examine what the structure of
  interrupts are in Linux, Windows, and FreeBSD, the general procedure of
  how they are handled, and where these interrupt handlers live in their
  respective OS.
  % Actually write an introduction, including why the reader should care
  \section{Linux}
  Linux has a fairly straightfoward way of handling interrupts. It has the interrupt
  structure of top and bottom halves, and each type of interrupt is registered
  to a unique handler in the kernel. The handler is defined by the driver, and
  is called when its unique handler ID appears.

  \subsection{Interrupts}
  Interrupts are how a piece of hardware communicates with the CPU, letting it
  know that it has some data ready for it. An interrupt must have an interrupt
  handler associated with it in order to have the communication handled properly.
  These interrupt handlers are contained in a devices driver. This is pretty
  universal across operating systems. What we are going to look at specifically
  is the structure of interrupts that Linux handles. \cite{robertlove2010}

  \subsection{Interrupt Structure}
  Interrupts are split into two parts in Linux: top halves, and bottom halves.
  These names are quite misleading as top halves and bottom halves are not halves
  at all. The top half is usually closer to a fifth or even an eighth. The reason
  for this is that interrupt handlers have to be very fast! Interrupts, as they
  are aptly named, stop everything the cpu is doing in order to communicate with
  hardware. When this interrupt is called, all the code in the top half of the
  interrupt has to be handled extremely quickly. Because of this, only time sensitive
  information can be stored in the top half, and all information in the top half
  is handled by the interrupt handler.

  The bottom half, on the other hand, can contain any extra information needed
  to process the request from the hardware. Because this information is not deemed
  time critical, it is queued up with the other requests that the CPU needs to handle
  and is put off until its time to be processed has come. \cite{robertlove2010}

  \subsection{Drivers}
  Drivers are where interrupt handlers exist in Linux. A handler ID is reserved
  by the system when a driver is installed. When an interrupt is called, the
  kernel goes and finds the appropriate handler, and calls the function to resolve
  the interrupt.

  Here is an example of how an interrupt handler is requested by a driver:
  \begin{lstlisting}
      if(request_irq(irqn, my_interrupt, IRQF_SHARED, "my_device", my_dev)) {
        printk(KERN_ERR "my_device: cannot register IRQ %d\n", irqn);
        return -EIO;
      }
  \end{lstlisting}

  In the above example, the requested interrupt line is \texttt{irqn}, the
  interrupt handler \texttt{my\_interrupt}, and the device \texttt{my\_device}.
  If the request is handled properly, it returns zero, otherwise it returns the
  code prints an error and returns an error code, \texttt{-EIO}, an IO error.
  \cite{robertlove2010}

  Next we'll look at how Windows handles these same ideas, and explore what
  interesting differences there are.

  \section{Windows}
  Windows has a similiar setup to Linux only in that drivers are where interrupt
  handlers are defined. There are some interesting differences in what device
  drivers are responsible for, and how fundamental system IO is handled in Windows
  as compared to Linux.
  \subsection{Interrupts}
  Windows has two different types of interrupts: general IO interrupts, and
  exceptions. Interrupts are asynchronus while exceptions are synchronus. Windows
  refers to catching these interrupts as traps, as they cause the kernel to differ
  from normal paths of execution. A trap is the equivilent of an interrupt handler.
  \cite{internals1}
  \subsection{Drivers}
  Drivers are important in every OS, but in Windows they have a very specific
  structure. Drivers have a few responsibilities besides handling interrupts as
  they do in Linux. They have the responsibility to start IO requests to the hardware,
  handle plug-and-play behavior, cancel IO routines, unload routines, and more.
  Drivers are much more of entire modules instead of specifically interrupt handlers
  as they are in Linux. \cite{internals2}

  This view of drivers makes the handling of IO much more of the drivers responsibility
  than it does in Linux. In Linux, the driver simply handles the interrupt and hands
  other information off to the OS. In Windows, the driver is more robust and has
  more responsibility on how things are communicated.

  Now that we've explored the basics of interrupts in Windows, we can look at
  FreeBSD, and see what similiarities and differences there are in its core structure.

  \section{FreeBSD}
  FreeBSD shares a very similiar interrupt structure and handling pattern with
  Linux. There are a few minute differences that should be pointed out.
  \subsection{Interrupts}
  Interrupts are handled almost identically to Linux's interrupt setup. Each
  interrupt must be registered via a device driver. FreeBSD has two types of
  interrupts: hardware interrupts and software interrupts. These are the equivalent
  of Linux's top and bottom halves. If a piece of information is time critical,
  it is put in a hardware interrupt, otherwise they are handed to a software
  interrupt. \cite{freebsd2016}
  \subsection{Interrupt Handler}
  Interestingly enough, interrupt handlers are also called traps in FreeBSD, the
  same as it is in Windows. Although they're name is the same, they handle fundamentally
  different structures. The FreeBSD interrupt handling model is almost identical
  to the Linux model. Interrupts are handled asynchronously, and have to be registered
  with a unique ID so that the kernel can find which driver needs to handle the
  interrupt. \cite{freebsd2016}

  \section{Conclusion}

  Interrupts are a key part of any OS. It is a fundamental in handling IO from
  external or internal devices. As apposed to other topics we've covered, it's
  interesting to see that the three operating systems have a lot in common in this
  area, as opposed to their individual unique approaches to other parts of the
  kernel.

  \clearpage
  \section{Appendix A - Linux Structs}
  All following structs have been pulled from the classroom text. \cite{robertlove2010}
  \begin{lstlisting}

    for future code samples

  \end{lstlisting}

  \section{Appendix B - Windows Structs}
  \begin{lstlisting}

    for future code samples
  \end{lstlisting}
  \section{Appendix C - FreeBSD Structs}
  \begin{lstlisting}
    for future code samples

  \end{lstlisting}

  \section{Bibliography}
%   \bibliographystyle{IEEEtran}
%   \bibliography{writing_3}
%
% \end{document}
