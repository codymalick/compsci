% \documentclass[10pt,letterpaper,onecolumn,draftclsnofoot]{IEEEtran}
% \usepackage[margin=0.75in]{geometry}
% \usepackage{listings}
% \usepackage{color}
% \usepackage{longtable}
% \usepackage{tabu}
\definecolor{dkgreen}{rgb}{0,0.6,0}
\definecolor{gray}{rgb}{0.5,0.5,0.5}
\definecolor{mauve}{rgb}{0.58,0,0.82}

\lstset{frame=tb,
  language=C,
  columns=flexible,
  numberstyle=\tiny\color{gray},
  keywordstyle=\color{blue},
  commentstyle=\color{dkgreen},
  stringstyle=\color{mauve},
  breaklines=true,
  breakatwhitespace=true,
  tabsize=4
}

% \begin{document}
\begin{titlepage}
  \title{CS 444 - Spring 2016 - Writing Assignment 1}
  \author{Cody Malick\\
  \texttt{malickc@oregonstate.edu}}
  \date{April 21, 2016}
  \maketitle
  \vspace*{4cm}
  \begin{abstract}
      \noindent Understanding how an operating system handles processes, threads,
      and CPU scheduling are important topics for an aspiring kernel developer to
      understand. In this report, we will cover how Linux, Windows, and FreeBSD
      handle these important tasks in their respective kernels, and contrast them.
  \end{abstract}
\end{titlepage}

\tableofcontents
\clearpage
\section{Introduction}
This report contains descriptions on how each of the operating systems, Linux,
Windows, and FreeBSD handle processes, threads, and CPU scheduling. A lot can be
learned about the design decisions of the minds behind the operating system by
how they handle these core constructs. We will start by describing how Linux
handles these, as Linux is the focus of this class.

\section{Linux}
  \subsection{Processes}
The first thing that should be defined, is simply what a process is: "A
\textit{process} is a program (object code stored on some media) in the midst of
execution."\cite{robertlove2010} Although the Linux kernel refers to processes
as tasks, in this report (for uniformity) we will only refer to them as processes.

A process in Linux is ultimately born, somewhere up the line, from the
\texttt{init} process. The \texttt{init} is the absolute root of every process
tree on any Linux machine. It has a process ID of 1. All processes are spawned
from either the \texttt{fork()} or \texttt{exec()} function calls. \texttt{Fork()}
creates an exact copy of the calling process as a child of the process, while
\texttt{exec()} loads a new executable into the process space and executes it.
These are the only ways new processes are created in Linux.

Each process has a process descriptor that describes every aspect of
the process, the \texttt{task\_struct}. The \texttt{task\_struct} is contained
in \texttt{<linux/sched.h>}. \texttt{Task\_struct} describes the current state
of the process, the parent process ID, the process ID, exit code, priority, etc.
Everything you could possibly want to know about a process in Linux is in that
struct. I've included the \texttt{Task\_struct} as Appendix A as it is a very
large structure.

Processes, of course, must have states of execution. On Linux, processes have
five states, each represented by flags in the \texttt{task\_struct}:

\begin{description}
  \item \texttt{TASK\_RUNNING}: The process is either running or in a run-queue
  ready to be run (run-queues will be discussed in the CPU scheduling section).
  \item \texttt{TASK\_INTERRUPTIBLE}: the process is sleeping (blocked) waiting
  for a condition to un-block. Once the condition is met, it transitions to
  \texttt{TASK\_RUNNING}.
  \item \texttt{TASK\_UNINTERRRUPTIBLE}: The same state as \texttt{TASK\_INTERRUPTIBLE}
  with the difference being that it does not wake once an interrupt is received.
  \item \texttt{\_TASK\_TRACED}: This flag shows that the process is being traced
  by another process. An example is GDB.
  \item \texttt{\_TASK\_STOPPED}: Process execution has halted and it is not
  possible to start running. This state is a result of a halting interrupt from
  the CPU, or if a debugger sends it a signal.
\end{description}

  \subsection{Threads}
As with processes, we will define what a thread is: "Threads of execution, often
shortened to \textit{threads}, are the objects of activity within the process."
\cite{robertlove2010} In Linux, threads are treated fundamentally the same as
processes. The only difference is that they have a shared address space,
filesystem resources, file descriptors, and signal handlers. Here is the
\texttt{thread\_info} struct:\cite{robertlove2010}

\begin{lstlisting}
  struct thread_info {
        struct task_struct    *task;
        struct exec_domain    *exec_domain;
        unsigned long         flags;
        unsigned long         status;
        __u32                 cpu;
        __s32                 preempt_count;
        mm_segment_t          addr_limit;
        struct restart_block  restart_block;
        unsigned long         previous_esp;
        __u8                  supervisor_stack[0];
};
\end{lstlisting}

Compared to the \texttt{task\_struct}, this is a realitively simple data structure.
It has a pointer to the \texttt{task\_struct} that is unique to that thread,
along with a few extra flags and sets of information that seperate threads from
processes.

  \subsection{CPU Scheduling}
Linux ships with a few different schedulers by default, including a real time
scheduler.\cite{redhat2016} The default scheduler for Linux is the CFS, or
the Completely Fair Scheduler.
The CFS is only fair in so far as it tries its best to split the processors time
into equal bits, 1/n, where n is the number of processes currently running. The
CFS weights this with process priority, or nice value, and the higher the weight
the larger percentage of processor time that process gets. So it's not completely
fair, but that is the idea behind it.

One of the major problems with this system is that a very large chunk of processes
creates incredibly small slices of CPU time. For example, if there were six-thousand
processes and/or threads that all had the same weight, then the cpu slice for
each process would be 1/6000. There is a point where that slice is not large
enough to even complete the context switch required to start executing the process
again. The CFS handles this by implementing a floor on slice size, known as
\textit{minimum granularity}.\cite{robertlove2010} With this system, every process
will always have a minimum amount of run time, but the system can become very
slow once this size is reached.

% https://www.kernel.org/doc/Documentation/scheduler/sched-design-CFS.txt
% This is a good resource for indepth analysis of the scheduler

\section{Windows}
Windows is a very different beast. The process and thread structure are quite a
bit different from Linux. The CPU scheduler Windows uses has the same basic idea
as the CFS, but varies greatly when it comes to how it handles priority and division
of CPU time.
  \subsection{Processes}
Processes in Windows are similiar to Linux in only one way: the both have a
process ID. Otherwise, processes in Windows handle very differently than Linux.
Here is the basic process struct in Windows, \texttt{\_PROCESS\_INFORMATION}:
\cite{msprocstruct2016}

\begin{lstlisting}
  typedef struct _PROCESS_INFORMATION {
    HANDLE hProcess;
    HANDLE hThread;
    DWORD  dwProcessId;
    DWORD  dwThreadId;
  } PROCESS_INFORMATION, *LPPROCESS_INFORMATION;
\end{lstlisting}

The \texttt{HANDLE} data type is a 32-bit unique identifier stored in the Windows
kernel to identify individual processes.\cite{mshandle2016} The \texttt{hThread}
points to the primary thread a process is executing on. These first two of these
identifying fields are used internally by the kernel to specify what functions
are performing operations on the process and thread objects. The second set of
identifiers, \texttt{dwProcessId} and \texttt{dwThreadId} are used as unique
identifiers for the process and thread themselves. These are the equivalent of
the PID used by the the Linux \texttt{tast\_struct}.

Process creation is also differnt in Windows. Instead of the constantly traceable
tree of child-parent relationships, a function called \texttt{CreateProcess()} is
invoked, resulting in a new process completely independent of the parent process.
\cite{msproccreate2016}

Process state in Windows is determined by thread state. Here are the possible
thread states in Windows:\cite{msthreadstate2016}
\begin{description}
  \item \texttt{Initialized}: A state that indicates the thread has been initialized,
  but has not yet started.
  \item \texttt{Ready}: A state that indicates the thread is waiting to use a processor
  because no processor is free. The thread is prepared to run on the next
  available processor.
  \item \texttt{Running}: A state that indicates the thread is currently using a processor.
  \item \texttt{Standby}: A state that indicates the thread is about to use a processor.
  Only one thread can be in this state at a time.
  \item \texttt{Terminated}: A state that indicates the thread has finished executing
  and has exited.
  \item \texttt{Transition}: A state that indicates the thread is waiting for a resource, other than
  the processor, before it can execute. For example, it might be waiting for its
  execution stack to be paged in from disk.
  \item \texttt{Unknown}: The state of the thread is unknown
  \item \texttt{Wait}: A state that indicates the thread is not ready to use the processor
  because it is waiting for a peripheral operation to complete or a resource to
  become free. When the thread is ready, it will be rescheduled.
\end{description}

This is interesting for a few reasons. First, there is an 'unknown' thread state.
In Linux, it is isn't possible for a thread or process to be in an unknown state,
only the ones listed. Next, it's interesting to see how Windows has so many more
states than Linux. One of the differences that sticks out most is that Windows
has a state when a process is queued to enter the processor in the \texttt{Standby}
state.
By having all these different states available, you can really see the real time
state of the process, as apposed to Linux where you have fewer states that cover
less situations. Then again, this could just be a result of Linux having better
processor and thread management, requiring less states.


  % https://msdn.microsoft.com/en-us/library/windows/desktop/ms681917(v=vs.85).aspx
  \subsection{Threads}
Threads, in Windows, are handled as subcomponents of processes. They are grouped
into thread pools, where multiple threads execute under one process. MSDN explains
the idea behind this structure is to reduce the number of application threads, and
provide management of worker threads. Threads are are spawned through the
\texttt{CreateThread()} function call.\cite{msthreadcreate2016}

% Here is the thread data structure:
% \begin{lstlisting}
%
% \end{lstlisting}

  \subsection{CPU Scheduling}
  The Windows CPU scheduler, by default, schedules processes in much the same way
  that the CFS schedules processes. A process is given the default weight, and
  the scheduler tries to divide the CPU time evenly among the running processes.
  Once we introduce weights or priority, the scheduler's behavior is very different.

  While the system is assigning CPU time slices to each process, if a higher priority
  thread becomes available, the CPU stops the current running process, mid-time
  slice, and starts running the higher priority threads.

  This is vastly different from the CFS in that regard. The CFS will allow a current
  running thread to complete it's time slice before moving it out of the processor.
  By introducing this kind of behavior, the power of priority states becomes glaringly
  clear.

  Windows has thirty-two levels of priority, each divided into subdivisions:
  \begin{description}
    \item \texttt{IDLE\_PRIORITY\_CLASS}
    \item \texttt{BELOW\_NORMAL\_PRIORITY\_CLASS}
    \item \texttt{NORMAL\_PRIORITY\_CLASS}
    \item \texttt{ABOVE\_NORMAL\_PRIORITY\_CLASS}
    \item \texttt{HIGH\_PRIORITY\_CLASS}
    \item \texttt{REALTIME\_PRIORITY\_CLASS}
  \end{description}

  Each of these classes has subdivisions inside of them, seven in each, that
  has steadily increasing higher priority, starting from one, rising to thirty-two.
  As a side note, the zero priority class is reserved for a thread that goes through
  and zeroes out memory when a process terminates.\cite{msschedule2016} Besides
  this large difference in priority, the two CPU schedulers are very similar.

\section{FreeBSD}
FreeBSD is an open source OS based off Unix, specifically the Berkeley variant,
BSD. It looks and feels similar to Linux in a few ways, but has some subtle
differences in process and thread structure. The scheduler is quite a bit different
from the CFS.
  \subsection{Processes}
Processes in FreeBSD are very similar to Linux's. Similiar to Linux's process
struct, \texttt{task\_struct}, the FreeBSD \texttt{proc} struct is large and
contains a lot of information. Processes are created using the same forking and
exec process that Linux uses. After a process is forked, a new process that is
identical, except for the PID, is created. The process has a copy of the parent's
resources and addresses space.

Process states are handled in a much more simplified manner than either Windows
or Linux in FreeBSD. Instead of having eight or five different states for a process,
it only has three:
\begin{description}
  \item \texttt{NEW}: Process is undergoing creation
  \item \texttt{NORMAL}: Thread or threads will be runnable, sleeping, or stopped
  \item \texttt{ZOMBIE}: The process is undergoing process termination
\end{description}

As per above, \texttt{NORMAL} embodies three subdivisions, runnable, sleeping,
or stopped, but overall the classification is simplified. \cite{kirkgeorgebsd}

  \subsection{Threads}
A notable difference between Linux and FreeBSD is that it does not treat threads
and processes the same way. Threads are handle in a more simliar fashion to Windows.
Each thread is attached to a parent process. If multiple threads are requested
for one process, new processes are spawned, but they have the same PID. When a
user looks for a process in the OS, they will only see the single entry representing
the parent process.

Thread states are dictated by the list in the processes section, but have a caviat.
When a thread is not running on the processor, which there can only be one of,
all threads are in one of three queues: the run queue, the sleep queue, or the
turnstile queue. As the names dictate, the run queue contains threads that are
in a runnable state. Threads that are blocked and are awaiting events are in the
sleep queue or the turnstile queue.

The turnstile queue is exclusively for short blocks on threads. In FreeBSD, short
blocks are exclusively the result of read/write blocks. Because these kinds of
blocks are common, each thread creates a turnstile queue when it is initialized,
and creates a list of all threads that are blocked because of a read/write block
on a certain piece of memory. When that block is lifted, the turnstile queue manages
the order in which the blocked threads are resolved. Allocating turnstile queues
on each thread as apposed to each lock results in lower memory usage by the kernel.

Sleep queues are similar to turnstiles in that they contain lists of threads
where blocking is occuring, but the sleep queue is exclusively for medium to long
term locks that can send interrupts if the lock is held for too long. \cite{kirkgeorgebsd}
  \subsection{CPU Scheduling}
The default CPU scheduler for FreeBSD is the ULE scheduler. ULE is not an acronym.
\cite{kirkgeorgebsd} The scheduler is contained in the namespace
\texttt{\/sys\/kern\/sched\_ule.c}, and if you remove the underscore in the namespace,
you will see the reasoning behind the name.

The ULE scheduler is split into two parts in FreeBS: a low-level scheduler, and
a high-level scheduler. The low-level scheduler runs often, and the high-level
scheduler runs a few times a second.

The low-level scheduler runs extremely often, whenever a thread is blocked. When
a block occurs, the low-level scheduler pulls the highest priority process from
a set of run queues. These queues are organized from low to high priority. The
high-level scheduler decides what the priority of each thread is, and sorts it
into the appropriate run queue. Each core has it's own set of these run queues
to prevent two cores from trying to access the same queue at the same time.

The different cores run a round robin system on the threads in the queues. Each
thread gets equal runtime in the form of time quantums. These time quantums are
the same as time slices. If a process uses an entire time quantum, it is moved
to the back of the the queue that it came from, and a context switch occurs to the
next highest priority thread.\cite{kirkgeorgebsd}
  % https://msdn.microsoft.com/en-us/library/windows/desktop/ms685100(v=vs.85).aspx
\clearpage
\section{Appendix A}
The \texttt{task\_struct}:\cite{linustorvalds2016}
\begin{lstlisting}
  struct task_struct {
/* these are hardcoded - don't touch */
  volatile long        state;          /* -1 unrunnable, 0 runnable, >0 stopped */
  long                 counter;
  long                 priority;
  unsigned             long signal;
  unsigned             long blocked;   /* bitmap of masked signals */
  unsigned             long flags;     /* per process flags, defined below */
  int errno;
  long                 debugreg[8];    /* Hardware debugging registers */
  struct exec_domain   *exec_domain;
/* various fields */
  struct linux_binfmt  *binfmt;
  struct task_struct   *next_task, *prev_task;
  struct task_struct   *next_run,  *prev_run;
  unsigned long        saved_kernel_stack;
  unsigned long        kernel_stack_page;
  int                  exit_code, exit_signal;
  /* ??? */
  unsigned long        personality;
  int                  dumpable:1;
  int                  did_exec:1;
  int                  pid;
  int                  pgrp;
  int                  tty_old_pgrp;
  int                  session;
  /* boolean value for session group leader */
  int                  leader;
  int                  groups[NGROUPS];
  /*
   * pointers to (original) parent process, youngest child, younger sibling,
   * older sibling, respectively.  (p->father can be replaced with
   * p->p_pptr->pid)
   */
  struct task_struct   *p_opptr, *p_pptr, *p_cptr,
                       *p_ysptr, *p_osptr;
  struct wait_queue    *wait_chldexit;
  unsigned short       uid,euid,suid,fsuid;
  unsigned short       gid,egid,sgid,fsgid;
  unsigned long        timeout, policy, rt_priority;
  unsigned long        it_real_value, it_prof_value, it_virt_value;
  unsigned long        it_real_incr, it_prof_incr, it_virt_incr;
  struct timer_list    real_timer;
  long                 utime, stime, cutime, cstime, start_time;
/* mm fault and swap info: this can arguably be seen as either
   mm-specific or thread-specific */
  unsigned long        min_flt, maj_flt, nswap, cmin_flt, cmaj_flt, cnswap;
  int swappable:1;
  unsigned long        swap_address;
  unsigned long        old_maj_flt;    /* old value of maj_flt */
  unsigned long        dec_flt;        /* page fault count of the last time */
  unsigned long        swap_cnt;       /* number of pages to swap on next pass */
/* limits */
  struct rlimit        rlim[RLIM_NLIMITS];
  unsigned short       used_math;
  char                 comm[16];
/* file system info */
  int                  link_count;
  struct tty_struct    *tty;           /* NULL if no tty */
/* ipc stuff */
  struct sem_undo      *semundo;
  struct sem_queue     *semsleeping;
/* ldt for this task - used by Wine.  If NULL, default_ldt is used */
  struct desc_struct *ldt;
/* tss for this task */
  struct thread_struct tss;
/* filesystem information */
  struct fs_struct     *fs;
/* open file information */
  struct files_struct  *files;
/* memory management info */
  struct mm_struct     *mm;
/* signal handlers */
  struct signal_struct *sig;
#ifdef __SMP__
  int                  processor;
  int                  last_processor;
  int                  lock_depth;     /* Lock depth.
                                          We can context switch in and out
                                          of holding a syscall kernel lock... */
#endif
};
\end{lstlisting}
\clearpage
\section{Appendix B}
FreeBSD \texttt{proc} struct: \cite{freebsdstruct2016}
\begin{lstlisting}
  /*
 * Description of a process.
 *
 * This structure contains the information needed to manage a thread of
 * control, known in UN*X as a process; it has references to substructures
 * containing descriptions of things that the process uses, but may share
 * with related processes.  The process structure and the substructures
 * are always addressable except for those marked "(PROC ONLY)" below,
 * which might be addressable only on a processor on which the process
 * is running.
 */
struct  proc {
        struct  proc *p_forw;           /* Doubly-linked run/sleep queue. */
        struct  proc *p_back;
        struct  proc *p_next;           /* Linked list of active procs */
        struct  proc **p_prev;          /*    and zombies. */

        /* substructures: */
        struct  pcred *p_cred;          /* Process owner's identity. */
        struct  filedesc *p_fd;         /* Ptr to open files structure. */
        struct  pstats *p_stats;        /* Accounting/statistics (PROC ONLY). */        struct  plimit *p_limit;        /* Process limits. */
        struct  vmspace *p_vmspace;     /* Address space. */
        struct  sigacts *p_sigacts;     /* Signal actions, state (PROC ONLY). */

#define p_ucred         p_cred->pc_ucred
#define p_rlimit        p_limit->pl_rlimit

        int     p_flag;                 /* P_* flags. */
        char    p_stat;                 /* S* process status. */
        char    p_pad1[3];

        pid_t   p_pid;                  /* Process identifier. */
        struct  proc *p_hash;    /* Hashed based on p_pid for kill+exit+... */
        struct  proc *p_pgrpnxt; /* Pointer to next process in process group. */
        struct  proc *p_pptr;    /* Pointer to process structure of parent. */
        struct  proc *p_osptr;   /* Pointer to older sibling processes. */

/* The following fields are all zeroed upon creation in fork. */
#define p_startzero     p_ysptr
        struct  proc *p_ysptr;   /* Pointer to younger siblings. */
        struct  proc *p_cptr;    /* Pointer to youngest living child. */
        pid_t   p_oppid;         /* Save parent pid during ptrace. XXX */
        int     p_dupfd;         /* Sideways return value from fdopen. XXX */

        /* scheduling */
        u_int   p_estcpu;        /* Time averaged value of p_cpticks. */
        int     p_cpticks;       /* Ticks of cpu time. */
        fixpt_t p_pctcpu;        /* %cpu for this process during p_swtime */
        void    *p_wchan;        /* Sleep address. */
        char    *p_wmesg;        /* Reason for sleep. */
        u_int   p_swtime;        /* Time swapped in or out. */
        u_int   p_slptime;       /* Time since last blocked. */

        struct  itimerval p_realtimer;  /* Alarm timer. */
        struct  timeval p_rtime;        /* Real time. */
        u_quad_t p_uticks;              /* Statclock hits in user mode. */
        u_quad_t p_sticks;              /* Statclock hits in system mode. */
        u_quad_t p_iticks;              /* Statclock hits processing intr. */

        int     p_traceflag;            /* Kernel trace points. */
        struct  vnode *p_tracep;        /* Trace to vnode. */

        int     p_siglist;              /* Signals arrived but not delivered. */

        struct  vnode *p_textvp;        /* Vnode of executable. */

        char    p_lock;                 /* Process lock (prevent swap) count. */
        char    p_pad2[3];              /* alignment */

/* End area that is zeroed on creation. */
#define p_endzero       p_startcopy

/* The following fields are all copied upon creation in fork. */
#define p_startcopy     p_sigmask

        sigset_t p_sigmask;     /* Current signal mask. */
        sigset_t p_sigignore;   /* Signals being ignored. */
        sigset_t p_sigcatch;    /* Signals being caught by user. */

        u_char  p_priority;     /* Process priority. */
        u_char  p_usrpri;       /* User-priority based on p_cpu and p_nice. */
        char    p_nice;         /* Process "nice" value. */
        char    p_comm[MAXCOMLEN+1];

        struct  pgrp *p_pgrp;   /* Pointer to process group. */

        struct  sysentvec *p_sysent; /* System call dispatch information. */

        struct  rtprio p_rtprio;        /* Realtime priority. */
/* End area that is copied on creation. */
#define p_endcopy       p_addr
        struct  user *p_addr;   /* Kernel virtual addr of u-area (PROC ONLY). */
        struct  mdproc p_md;    /* Any machine-dependent fields. */

        u_short p_xstat;        /* Exit status for wait; also stop signal. */
        u_short p_acflag;       /* Accounting flags. */
        struct  rusage *p_ru;   /* Exit information. XXX */
};
\end{lstlisting}
\section{Bibliography}

% \bibliographystyle{IEEEtran}
% \bibliography{writing_1}

% \end{document}
