\documentclass[10pt,letterpaper]{article}

\usepackage[margin=0.75in]{geometry}

\begin{document}

  \title{Weekly Summary 2}
  \author{Cody Malick\\
  \texttt{malickc@oregonstate.edu}}
  \date{\today}
  \maketitle

    In Robert Love's third and fourth chapters, \textit{Proccess Management}
    and \textit{Proccess Scheduling}(\date{2010}), Love explains that Linux uses
    processes and schedulers to manage tasks and how they are executed on a
    machine. Love does this by explaining the fundemental structure of a process
    , how a scheduler work with those processes, and the pros and cons of
    different scheduler implementations. Love explains these principles in order
    to educate the user on the basic working structure of the kernel, in order
    to further enlighten the reader on the kernel's functionality. The in-depth
    explainations and definitions of different parts of the proccess structure,
    as well as the scheduler implementations, imply that Love is writing for
    aspiring kernel developers.

\end{document}
