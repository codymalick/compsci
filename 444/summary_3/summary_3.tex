\documentclass[10pt,letterpaper]{article}

\usepackage[margin=0.75in]{geometry}

\begin{document}

  \title{Weekly Summary 3}
  \author{Cody Malick\\
  \texttt{malickc@oregonstate.edu}}
  \date{April 15, 2016}
  \maketitle

    In Robert Love's fourteenth chapter, \textit{The Block I/O Layer}
    (\date{2010}), Love explains that Linux stores information using block
    devices and writes and reads data efficiently using I/O schedulers.
    Love does this by explaining the fundamental structure of a block device,
    how Linux abstracts the disk using buffer heads and the bio structure, and
    how a variety of I/O schedulers work with request queues to manage I/O
    requests. Love's purpose for explaining these constructs is to educate the
    user on the workings of block device I/O used in the kernel in order to
    further enlighten the reader on the kernel's functionality.
    The in-depth explanations and definitions of different parts of block
    devices and I/O scheduler implementations, imply that Love is writing for
    aspiring kernel developers.

\end{document}
